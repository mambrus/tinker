
\part{Using a pThreads, mqueue based kernel}
%------------------------------------------------------------------------------
%------------------------------------------------------------------------------
%------------------------------------------------------------------------------
\chapter{Introduction}
In this part, we'll start using TinKer by playing with some examples. That the examples are targeting concurrency issues. Real-time issues are a very often tightly related to concurrency\footnote{An aspect of concurrency}. But remember, formally it's not the same.
\\\\
TinKer follows the POSIX 1003.1c \& 1003.1b standards. The first one covers concurrency but real-time is not mentioned at all. In the second standard real-time is mentioned, but the meaning is not complete and addresses queues only.
\\\\
TinKer covers POSIX 1003.1c, but does it so that it grants real-time behaviour to the system that uses it. The end result is a real-time kernel with a standardised, very widespread and very well defined API.
\\\\
All the examples have been verified against two other independent OS's to verify their POSIX 1003.1c and 1003.1b compliance. 
\\\\
The examples will be tailored for a GNU tool-chain. Other environment for TinKer exists\footnote{MSVC, Keil, Borland} however.
\\\\
This part could be extended to cover the subject of concurrent programming and real-time of it's own, which is not the purpose. Literature are full of good books in these subjects. TODO references \ref{FIXME}

%------------------------------------------------------------------------------
%------------------------------------------------------------------------------
%------------------------------------------------------------------------------
\chapter{Building the kernel}
Start a shell and move to the directory where you've the TinKer sources. Now execute the following:
\\\\
\textit{make configure}
\\\\
This is a boot-strap build and will prepare the project to be configured. It has to be run once after each check-out or fresh source package installation. After that, you don't need to reinvoke this command even if you configure and build the kernel for different targets (or any other option changes).
\\\\
Then run the configure script with the appropriate parameters. You can do this either from the source directory root or from an empty directory\footnote{Same principle as most Open Source projects}. For example\footnote{Type  \textit{./configure --help} for a full list of usage and options.}:
\\\\
\textit{./configure -C --host=arm-hixs-elf MCPU=arm7tdmi BOARD=lpc21xx}
\\\\
This will configure the project tor the target architecture ARM\footnote{1'st part of the \textit{host} name triplet defines the architecture, 2'nd defines the system interface and 3'rd defines the ABI}, using the system interface HIXS and calling convention used by ELF. The specific CPU variant arm7tdmi\footnote{See gcc documentation for a list of valid CPU's} is pointed out and the BSP board adaption is lpc21xx.
\\\\
Now all you have to do is:
\\\\
\textit{make all}
\\\\
\textit{make install}
\\\\
This will build the kernel as a library\footnote{libtinker.a} and install this and Tinker's header files on your build system\footnote{Note, to install on your target you still need to statically link with an application and download the binary}.

\paragraph{Removal}
In case you need to remove TinKer from your system, invoke the command: \textit{make cleanall}\\\\
In case you only need to remove intermediate files in your build directory invoke: \textit{make clean}\\\\

\paragraph{When defaults are not enough}
What is built as default \textit{differs} from target to target, depending on what the maintainer of that target felt was reasonable considering it's size and performance. You might want to enter the following configure options to make sure you'll get a build that will work for the examples:
\\\\
\textit{./configure -C --host=\$arch-ven-abi\footnote{For example: \textit{powerpc-eabisim}} MCPU=\$your\_cpu\footnote{For example: \textit{860}} BOARD=\$your\_board\footnote{For example: \textit{generic}} --enable-itc --disable-ptimer --enable-posix\_rt --enable-pthread --enable-kmem=20000 --enable-max\_prio=0x10 --enable-max\_threads=500 --enable-norm\_stack=0x0800}




%------------------------------------------------------------------------------
%------------------------------------------------------------------------------
%------------------------------------------------------------------------------
\chapter{Building your applications}
Whence you have built and installed the kernel, it's now part of your tool-chains installed structure. The kernel is in the form of a library\footnote{libtinker.a} with corresponding headers. You can have several TinKer builds installed on each tool system, one for each part of the canonical name triplet\footnote{Fore example one for each of the targets: arm-hixs-elf, arm-elf, bfin-hixs-elf, powerpc-hixs-eabi \& powerpc-eabisim}.

TinKer gets installed as in the structure in table~\ref{inslall_structure}:

% ./powerpc-eabisim/lib/libtinker.a
% ./lib/gcc/arm-hixs-elf/4.1.1/tinker/crt0.o
% ./powerpc-eabisim/include/tinker/pthread.h


\begin{table}[!hbp]

\begin{verbatim}
PREFIX/
    arch-ven-abi/
        lib/
           libtinker.a
           tinker/
             flash\_gnu.ld
             ram\_gnu.ld
       include/
          tinker/
             *.h
    lib/
       gcc/
          arch-ven-abi/
             x.y.z/
                tinker/
                   crt0.o
\end{verbatim}
\caption{Where TinKer files get installed}\label{inslall_structure}
\end{table}

Due to TinKer's ambition to cope with different libc, it also does not impose any changes or requirements on the headerfiles. Note that libc usually allready has several headerfiles with the same name as TinKer's headers, but with different or incompatible contents.

The TinKer headers are designed to be able to not only cope, but also take advantage of certain parts of any existing headerfiles. It's therefore very important that you application builds headerfile search-path finds the TinKer headers first and the original headerfiles after.

TinKer will aid you in this task by the use of the mandatory define CHAINPATH. Use this define instead of a searchpath to the original headerfiles. It will make it possible for TinKer sources to overlay header files\footnote{I.e. either as whole file substitution or as combined with TinKer's equivalent}. automatically (internally handled by macro expansion).

Just set this define to what the system header-files would otherwise be, and set the \textit{-I \$dir} to TinKer header search path\footnote{See table~\ref{inslall_structure}}.
\\\\
To compile a simple hello.c program, follow the steps\footnote{Put them in a script or makefile} as in table~\ref{build_hello} (please note that the order of each file in the linking part is relevant):
\begin{table}[!hbp]
\begin{verbatim}
export SYSH=/prefix/arch-ven-abi/include
export TINKI=$SYSH/tinker
arch-ven-abi-gcc -c -O0 -g3 -gstabs -DCHAINPATH=$SYSH -I $TINKI hello.c \
   -o hello.o

export SYSL=/prefix/arch-ven-abi/lib
export GCCL=/prefix/lib/gcc/arch-ven-abi/x.y.z
arch-ven-abi-ld -o hello.abi -L$SYSL -L$GCCL  $GCCL/tinker/crt0.o -ltinker \
   hello.o -Ttinker/flash_gnu.ld -lc -lm -lgcc

arch-ven-abi-objcopy -O ihex hello.abi hello.hex
arch-ven-abi-objdump -d hello.abi > hello.objdump
\end{verbatim}
\caption{Build steps: hello.c}\label{build_hello}
\end{table}

I'd recommend that you build and test the examples in this whole part of the text in the following three cases one by one\footnote{I.e. build ALL the examples for each of the mentioned cases one by one. The order is from the easiest to the hardest to get started with.}:

\begin{itemize}
\item As native programs using another OS available to you, for example Linux \footnote{You'll need a Linux kernel 2.6.8 or newer} will work fine. Note that this build does not use TinKer at all and the build steps are also different\footnote{For example: \textit{gcc -O2 -g -o hello hello.c}}. It's for your exercise, to remember to always make certain your applications are portable across targets.
\item Build the kernel for your desktop computer as a target, and link your applications towards this library. This will enable you to run the examples in a simulated environment \footnote{Available simulated targets are Linux x86 and Cygwin (Win32)} which is more convenient than to run it on a dedicated target.
\item Cross build \& run for a dedicated OS-less target and remote debug it. 
\end{itemize}
%
The build process regarding the last point is identical to the simulated version, except that code needs to be downloaded and cross tools need to be available. If you're unconfident about how to set up a GNU cross tool-chain or you don't have a supported target available, the simulated environmen will work just as fine.
\\\\

%------------------------------------------------------------------------------
%------------------------------------------------------------------------------
%------------------------------------------------------------------------------
\chapter{"Hellow world"}
Let's start off by a classical hello world example. The program consists of a main part in table~\ref{hello1} and one thread in table~\ref{hello_thread}.

The initial part in table~\ref{top_of_app} is just a couple for includes and defines for your convenience. Please reuse it for the rest of the examples.

\begin{table}[!hbp]
\begin{verbatim}
#include <stdio.h>
#include <stdlib.h>
#include <pthread.h>
#include <assert.h>
#include <string.h>
#include <errno.h>

#ifndef TRUE
   #define TRUE 1
#endif
#ifndef FALSE
   #define FALSE 0
#endif
#if !defined(TINKER) && defined(_WIN32) &&  defined(_MSC_VER)
   #include <windows.h>
   #define sleep(x) (Sleep(x * 1000))
   #define usleep( x ) (Sleep( (unsigned long)x / 1000ul ) )
#endif
\end{verbatim}
\caption{Top of your source - reuse this in all applications that follow.\label{top_of_app}}
\end{table}

\begin{table}[!hbp]
\begin{verbatim}
void *hello(void *arg){
   int i;

   for (i=0; i<10; i++){
      printf(%s,(char*)arg);
      msleep(1000);
   }
}
\end{verbatim}
\caption{The "Hello" thread.\label{hello_thread}}
\end{table}

\begin{table}[!hbp]
\begin{verbatim}
int main(char argc, char **argv)
{
   pthread_t         threadA;

   printf("Hello Universe\n");
   printf("Root started\n");

   assert (pthread_create(&threadA, NULL, hello, "Hello World!\n" ) == 0);

   assert (pthread_join(threadA, NULL) == 0);

   return 0;
}
\end{verbatim}
\caption{Hello world program (TinKer style).\label{hello1}}
\end{table}
A nice twist to exemplify the difference between a thread and it's body is modifying the program in table~\ref{hello1} to the one in table~\ref{hello2} to let several threads use the same body. Notice that each thread will have it's own copies of all the variables defined as local in that body.

\begin{table}[!hbp]
\begin{verbatim}
int main(char argc, char **argv)
{
   pthread_t         threadA,threadB;

   printf("Hello Universe\n");
   printf("Root started\n");

   assert (pthread_create(&threadA, NULL, hello, "Hello " ) == 0);
   assert (pthread_create(&threadB, NULL, hello, "World!\n" ) == 0);

   assert (pthread_join(threadA, NULL) == 0);
   assert (pthread_join(threadB, NULL) == 0);

   return 0;
}
\end{verbatim}
\caption{Simple modification of Hello World. Two threads use the same body.\label{hello2}}
\end{table}

%------------------------------------------------------------------------------
%------------------------------------------------------------------------------
%------------------------------------------------------------------------------
\chapter{A mutex example}
This example shows the usage of a mutex. Please notice that the mutex variable in this example has to be public to both threads\footnote{I.e. it has to be defined in the code section} and it therefore must be defined before both threads.

\begin{table}[!hbp]
\begin{verbatim}
pthread_mutex_t glob_mux = PTHREAD_MUTEX_INITIALIZER;

void *funcA(void *arg){
   int i,j;

   for (i=0; i<6; i++){
      pthread_mutex_lock(&glob_mux);
      for (j=0;j<20;j++){
         printf("Thread A - In critical section [%d, %d]\n",i,j);
         usleep(200000);
      }
      pthread_mutex_unlock(&glob_mux);
   }
   return arg;
}

void *funcB(void *arg){
   int i,j;

   for (i=0; i<4; i++){
      pthread_mutex_lock(&glob_mux);
      for (j=0;j<20;j++){
         printf("--[%d, %d]--------->>Thread B<< is king!\n",i,j);
         usleep(300000);
      }
      pthread_mutex_unlock(&glob_mux);
   }
   return arg;
}
\end{verbatim}
\caption{Two threads - one critical section.\label{mutex_threads}}
\end{table}

\begin{table}[!hbp]
\begin{verbatim}
int main(char argc, char **argv)
{
   pthread_t         threadA, threadB;   

   assert (pthread_create(&threadA, NULL, funcA, (void*)1 ) == 0);
   assert (pthread_create(&threadB, NULL, funcB, (void*)2 ) == 0);

   assert (pthread_join(threadA, NULL) == 0);
   assert (pthread_join(threadB, NULL) == 0);

   return 0;
}
\end{verbatim}
\caption{Mutex program (TinKer style).\label{mutex1}}
\end{table}
%------------------------------------------------------------------------------
%------------------------------------------------------------------------------
%------------------------------------------------------------------------------
\chapter{First look at queues}

\begin{table}[!hbp]
\begin{verbatim}
void *thread1(void *inpar ){ 
   int         loop_cntr            = 0; 
   int         loop_cntr2           = 0; 
   mqd_t       q; 
   int         rc; 
   char        msg_buf[16]; 

   q = mq_open( QNAME, /*O_NONBLOCK | */O_WRONLY, 0 ,NULL); 
   if (q==(mqd_t)-1){ 
      int myerrno = errno;

      printf("1.Errno = %d\n",myerrno);
      if (myerrno == EACCES)
         printf("EACCES\n");

      perror(strerror(myerrno)); 
      assert("Queue opening for writing faliure" == NULL);
   } 

   while (TRUE) {
      usleep(100000); 
      loop_cntr++; 
      if (loop_cntr>=2){ 
         loop_cntr2++; 
         loop_cntr = 0; 

         msg_buf[0] = loop_cntr2;
         printf("sending....\n"); 
         rc = mq_send(q, msg_buf, MYMSGSIZE, 5); 
         if (rc==(mqd_t)-1){
            int myerrno = errno;

            printf("2.Errno = %d\n",myerrno);
            if (myerrno == EACCES)
               printf("EACCES\n");

            perror(strerror(myerrno));
            assert("Queue writing faliure" == NULL); 
         } 
         printf("sent!\n"); 
      } 
   } 
   return (void*)1; 
} 
\end{verbatim}
\caption{Queue thread 1.\label{q_thread1}}
\end{table}

\begin{table}[!hbp]
\begin{verbatim}
void *thread2(void *inpar){ 
   char        msg_buf[16]; 
   mqd_t       q; 
   int         rc; 

   q = mq_open( QNAME, O_RDONLY, 0 ,NULL); 
   if (q==(mqd_t)-1){ 
      int myerrno = errno;

      printf("3.Errno = %d\n",myerrno);
      if (myerrno == EACCES)
         printf("EACCES\n");
      perror(strerror(myerrno)); 

      assert("Queue opening for reading faliure" == NULL); 
   } 

   while (TRUE) {
      printf("receiving....\n");
      rc = mq_receive(q, msg_buf, MYMSGSIZE, NULL);
      if (rc==(mqd_t)-1){
         int myerrno = errno;

         printf("4.Errno = %d\n",myerrno);
         if (myerrno == EACCES)
            printf("EACCES\n");

         perror(strerror(myerrno)); 
         assert("Queue reading faliure\n" == NULL); 
      } 
      printf("Received: %d of length %d\n",msg_buf[0],rc); 
   } 
   return (void*)1; 
}
\end{verbatim}
\caption{Queue thread 2.\label{q_thread2}}
\end{table}


\begin{table}[!hbp]
\begin{verbatim}
int main(char argc, char **argv)
{ 
   pthread_t T1_Thid,T2_Thid; 
   mqd_t q2;
   int loop =0;   
   struct mq_attr qattr;

   printf("Unlinking old queue name (if used). \n");  
   mq_unlink(QNAME);  //Don't assert - "failiure" is normal here
   sleep(1);

   qattr.mq_maxmsg = 3;
   qattr.mq_msgsize = MYMSGSIZE;
   q2 = mq_open( QNAME,O_CREAT|O_RDWR,0666,&qattr);  

   if (q2==(mqd_t)-1){      
      int myerrno = errno;

      printf("Errno = %d\n",myerrno);
      if (myerrno == EACCES)
         printf("EACCES\n");
      perror(strerror(myerrno)); 
      assert("Queue opening for creation faliure" == NULL); 
   }
   sleep(3);
   printf("Queues created\n"); 

   assert (pthread_create(&T1_Thid, NULL, thread1, 0) == 0); 
   assert (pthread_create(&T2_Thid, NULL, thread2, 0) == 0); 

   printf("Root started\n"); 
   for (loop=0;loop<10;loop++) {
      printf("Root 2s bling!!!!!!!!!!!!!!!!!\n");      
      sleep(2); 
   } 
   printf("Root is done\n");

   printf("Closing queue handle. Both threads should still work <------------\n");  
   assert (mq_close(q2) == 0);
   sleep(10);

   printf("Unlinking queue name. Threads should block now <------------------\n");    
   assert(mq_unlink(QNAME) == 0);
   sleep(5);

   printf("Hmm, does Linux really following standard? <----------------------\n");  
   sleep(5);

   printf("Killing thread 3\n");  
   assert (pthread_cancel(T3_Thid) == 0);

   printf("Killing thread 2\n");  
   assert (pthread_cancel(T2_Thid) == 0);
   return 0;
} 
\end{verbatim}
\caption{Main program - queues example.\label{q_main}}
\end{table}
%------------------------------------------------------------------------------
%------------------------------------------------------------------------------
%------------------------------------------------------------------------------
\chapter{Master-worker programming model - sorting}
This example implements quick-sort as a threaded operation in a master-worker programming model. In principle it's the same algorithm as the text-book qsort, except that instead of using recursion new threads are started.

The principle is that each word in the text will get it's own thread. This thread only job is to move the word into position and then exit.

It's a silly way to implement qsort on a single-core system\footnote{On a multi-core system it could however actually be quite effective!}. The example only serves the purpose of exemplifying master-worker threading model

Depending on how balanced the text to be sorted is, the number of worker threads can be up to $(2 * n) - 1$, where $n$ is the number of words in the text to be sorted\footnote{Be carefull about the stack usage!}.

\begin{table}[!hbp]
\begin{verbatim}
#define MAX_WORDS 500
#define MAX_SORT_ELEMENT_SIZE 255  //!< Limitation of each sorting elements size

typedef int my_comparison_fn_t (  
   const void *L,  //!< <em>"Leftmost"</em> element to compare with
   const void *R   //!< <em>"Rightmost"</em> element to compare with
);

typedef my_comparison_fn_t *cmpf_t;

typedef struct myarg_t{
   void *a;              //!< The array to be sorted
   int l;                //!< left index
   int r;                //!< right index
   int sz;               //!< Size of each element
   cmpf_t cmpf;          //!< Comparison function 
}myarg_t;

void my_swap (
   void *a,              //!< The array to be sorted
   int l,                //!< left index
   int r,                //!< right index
   int sz                //!< Size of each element   
){
   char t[MAX_SORT_ELEMENT_SIZE];
   memcpy(t,                  &((char*)a)[l*sz],   sz);
   memcpy(&((char*)a)[l*sz],  &((char*)a)[r*sz],   sz);
   memcpy(&((char*)a)[r*sz],  t,                   sz);
}

unsigned int my_qsort_depth = 0; 
unsigned int my_curr_depth = 0;  

#define ISALFANUM( expr ) (\
   (expr >= 'A') && (expr <= 127) ? 1 : 0  \
)

int my_strvcmp (  
   const void *L,
   const void *R 
){
   return strncmp(*(char**)L,*(char**)R,80);   
}
\end{verbatim}
\caption{Structures and data.\label{ccsort_structs}}
\end{table}

\begin{table}[!hbp]
\begin{verbatim}
static const char theText[] = "                                     \
AROUND THE WORLD IN EIGHTY DAYS                                     \
                                                                    \
Chapter I                                                           \
                                                                    \
IN WHICH PHILEAS FOGG AND PASSEPARTOUT ACCEPT EACH OTHER,           \
THE ONE AS MASTER, THE OTHER AS MAN                                 \
                                                                    \
Mr. Phileas Fogg lived, in 1872, at No. 7, Saville Row, Burlington  \
Gardens, the house in which Sheridan died in 1814.  He was one of   \
the most noticeable members of the Reform Club, though he seemed    \
";
\end{verbatim}
\caption{The text to sort. Cut \& paste aprox. 100-300 words from any text of your choice.\label{ccsort_text}}
\end{table}

\begin{table}[!hbp]
\begin{verbatim}
void *conc_quicksort ( void* inarg_vp){
   int i,j;
   void *v;
   struct myarg_t *inarg_p = (myarg_t *)inarg_vp;
   struct myarg_t leftArg;
   struct myarg_t rightArg;
   pthread_t   leftID;
   pthread_t   rightID;

   my_curr_depth++;
   if (my_curr_depth > my_qsort_depth)
      my_qsort_depth=my_curr_depth;

   if ( inarg_p->r > inarg_p->l ){

      v = (void**)(((char*)(inarg_p->a))+(inarg_p->r*inarg_p->sz));
      i = inarg_p->l-1; j = inarg_p->r;
      for (;;){
         while ( inarg_p->cmpf(
            (void**)(((char*)(inarg_p->a))+(++i*inarg_p->sz)), v ) < 0);
         while ( inarg_p->cmpf(
            (void**)(((char*)(inarg_p->a))+(--j*inarg_p->sz)), v ) > 0) ;
         if (i >= j) break;
         my_swap(inarg_p->a,i,j,inarg_p->sz);
      }
      my_swap(inarg_p->a,i,inarg_p->r,inarg_p->sz);

      leftArg.a   = inarg_p->a;
      leftArg.l   = inarg_p->l;
      leftArg.r   = i-1;
      leftArg.cmpf = inarg_p->cmpf;
      leftArg.sz  = inarg_p->sz;

      rightArg.a  = inarg_p->a;
      rightArg.l  = i+1;
      rightArg.r  = inarg_p->r;
      rightArg.cmpf= inarg_p->cmpf;
      rightArg.sz = inarg_p->sz;

      assert (pthread_create(&leftID,  NULL, conc_quicksort, &leftArg)  == 0);
      assert (pthread_create(&rightID, NULL, conc_quicksort, &rightArg) == 0);
      assert (pthread_join(leftID,  NULL) == 0);     
      assert (pthread_join(rightID, NULL) == 0);     
   }
   my_curr_depth--;
   return (void*)0;
}
\end{verbatim}
\caption{Body of each sorting thread (all threads use the same).\label{ccsort_thread}}
\end{table}

\begin{table}[!hbp]
\begin{verbatim}
int main(char argc, char **argv)
{ 
   pthread_t         sortID;
   unsigned int      i,j,k;
   int               inText = 0;
   struct myarg_t    sortArg;

   //Buid up the pointer table
   for (i=0,j=0;i<sizeof(theText);i++){
      assert (j<MAX_WORDS);
     if ( (inText == FALSE) && ISALFANUM(theText[i]) ){
         inText = 1;
         text_p[j] = (char *)&theText[i];
         j++;
      }

      if ( (inText == TRUE) && !ISALFANUM(theText[i]) ){
         inText = 0;
      }
   }

   sortArg.a   = text_p;
   sortArg.l   = 0;
   sortArg.r   = j-2;
   sortArg.sz  = sizeof(char*);
   sortArg.cmpf= my_strvcmp;

   assert (pthread_create(&sortID, NULL, conc_quicksort, &sortArg) == 0);
   assert (pthread_join(sortID, NULL) == 0);

   for (i=0,k=0;i<j;i++){
      assert (j<5000);
      for (k=0;ISALFANUM(text_p[i][k]);k++)
         putchar(text_p[i][k]);
      printf("\n");
   }
   return 0;
}
\end{verbatim}
\caption{Main program for the sorting example.\label{ccsort_main}}
\end{table}

%------------------------------------------------------------------------------
%------------------------------------------------------------------------------
%------------------------------------------------------------------------------
\chapter{Conditional variables}
A text-book example of how to use conditional variables

Two threads are updating the shared resource "count". When count reaches
the agreed upon threshold, a third thread acting as "listener" (and who
is blocked at that time) will be signalled and awakened.

\begin{table}[!hbp]
\begin{verbatim}
int count = 0;
pthread_mutex_t count_mutex = PTHREAD_MUTEX_INITIALIZER;
pthread_cond_t  count_threshold_cv = PTHREAD_COND_INITIALIZER;
#define WATCH_COUNT 12
#define TCOUNT 10

void *watch_count(void *idp){
   pthread_mutex_lock(&count_mutex);
   while (count < WATCH_COUNT){
      pthread_cond_wait(&count_threshold_cv, &count_mutex);
      printf("watch_count(): Thread %d, Count is %d\n", *(int*)idp, count);
   }
   pthread_mutex_unlock(&count_mutex);
   printf("watch_count(): Thread %d finished!\n", *(int*)idp, count);
   return (void*)0;
}
\end{verbatim}
\caption{Listener thread.\label{cond_watcher}}
\end{table}

\begin{table}[!hbp]
\begin{verbatim}
void *inc_count(void *idp){
   int i;

   for (i=0; i<TCOUNT; i++){
      pthread_mutex_lock(&count_mutex);
      count++;
      printf("inc_count(): Threads %d, old count %d, new count %d\n",
         *(int*)idp, count -1, count);
      if (count == WATCH_COUNT)
            pthread_cond_signal(&count_threshold_cv);
      pthread_mutex_unlock(&count_mutex);
      usleep(*(int*)idp * 700 + 1000000);
   }
   printf("inc_count(): Thread %d finished!\n", *(int*)idp, count);
   return (void*)0;
} 
\end{verbatim}
\caption{The writer threads.\label{cond_writer}}
\end{table}

\begin{table}[!hbp]
\begin{verbatim}
int thread_ids[3] = {0, 1, 2};
pthread_t            threads[3]; 
int main(char argc, char **argv)
{ 
   int i;

   printf("Hello world or TinKer targets\n");
   printf("Root started\n");

   assert (pthread_create(&threads[2], NULL, watch_count, &thread_ids[2] ) == 0);
   assert (pthread_create(&threads[0], NULL, inc_count, &thread_ids[0] ) == 0);
   assert (pthread_create(&threads[1], NULL, inc_count, &thread_ids[1] ) == 0);

   for (i=0; i<3; i++){
      assert (pthread_join(threads[i], NULL) == 0);
   }      
   return 0;   
}
\end{verbatim}
\caption{Main program - Conditional variables.\label{cond_main}}
\end{table}

%------------------------------------------------------------------------------
%------------------------------------------------------------------------------
%------------------------------------------------------------------------------
\chapter{Read-write locks}
A text-book example of how to use RW locks
\\\\
Three threads are reading a resource and two threads are writing it

\begin{itemize}
	\item Readers read more often then the writers write (which would be the most common real case).
	\item Readers will not block each other, but would block a writer
	\item A writer will block readers and writers
	\item When releasing lock $\Rightarrow$ If writer blocks readers and writers of same priority, release writer first (part of standard)
	\item When taking read lock $\Rightarrow$ If a writer is blocked and of higher priority than you, back of and let the writer have a chance on the lock.
\end{itemize}

\begin{table}[!hbp]
\begin{verbatim}
int count = 0;
#if defined(linux) &&  (!defined(TINKER))
struct pthread_rwlock_t_ *rw_lock;
#else
pthread_rwlock_t rw_lock = PTHREAD_RWLOCK_INITIALIZER;
#endif
#define TCOUNT 10

void *consumer(void *inpar){
   int done=0;
   int myCount=0;

   printf("<--Consumer thread [%d] starts\n",*(int*)inpar);
   while (myCount<(TCOUNT*2)){
      pthread_rwlock_rdlock(&rw_lock);
         printf("<--Consumer thread [%d] starts reading resource\n",
            *(int*)inpar);
         //Simulate slow access to a resource
         usleep(*(int*)inpar * 300 + 1000000);
         myCount = count;
         printf("<--Consumer thread [%d] finished reading resource.");
         printf("Value is= %d\n",*(int*)inpar,myCount);
      pthread_rwlock_unlock(&rw_lock);
      usleep(*(int*)inpar * 300 + 1000000);
   }
   printf("<--Consumer thread [%d] exits\n",*(int*)inpar);
   return (void*)0;
}

\end{verbatim}
\caption{Reader thread(s).\label{rwl_threads1}}
\end{table}

\begin{table}[!hbp]
\begin{verbatim}
void *producer(void *inpar){
   int i;
   printf("-->Producer thread [%d] starts\n",*(int*)inpar);
   for (i=0; i<TCOUNT; i++){
      pthread_rwlock_wrlock(&rw_lock);
         printf("-->Producer thread [%d] starts updating resource\n",
            *(int*)inpar);
         //Simulate slow access to a resource
         usleep(*(int*)inpar * 700 + 1000000); 
         count++;
         printf("-->Producer thread [%d] finished updating resource.); 
         printf("Value is= %d\n",*(int*)inpar,count);
      pthread_rwlock_unlock(&rw_lock);
      usleep(*(int*)inpar * 700 + 1000000);
   }
   printf("-->Producer thread [%d] exits\n",*(int*)inpar);
   return (void*)0;
} 

\end{verbatim}
\caption{Writer threads(s).\label{rwl_threads2}}
\end{table}


\begin{table}[!hbp]
\begin{verbatim}
int thread_ids[5] = {0, 1, 2, 3, 4};
pthread_t            threads[5]; 
int main(char argc, char **argv)
{ 
   int i;

   #if defined(linux) &&  (!defined(TINKER))
   pthread_rwlock_init(&rw_lock,NULL);
   #endif

   //Creating the reader threads
   assert (pthread_create(&threads[0], NULL, consumer, &thread_ids[0] ) == 0);
   assert (pthread_create(&threads[1], NULL, consumer, &thread_ids[1] ) == 0);
   assert (pthread_create(&threads[2], NULL, consumer, &thread_ids[2] ) == 0);

   //Creating the writer threads
   assert (pthread_create(&threads[3], NULL, producer, &thread_ids[3] ) == 0);
   assert (pthread_create(&threads[4], NULL, producer, &thread_ids[4] ) == 0);

   for (i=0; i<5; i++){
      assert (pthread_join(threads[i], NULL) == 0);
   }
   return 0;
}

\end{verbatim}
\caption{Main program - R/W locks example.\label{rwl_main}}
\end{table}


%------------------------------------------------------------------------------
%------------------------------------------------------------------------------
