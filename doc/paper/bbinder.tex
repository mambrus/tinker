\documentclass[a4paper,10pt]{book}
\frontmatter
\usepackage{dottex}

\begin{document}
%@MastersThesis{,
%author = {Michael Ambrus},
%title = {Embedded system specialist},
%school = {MIUN (Mid University of Sweden)},
%year = {2006},
%OPTkey = {Computer Sciense},
%OPTtype = {Kernel technology},
%OPTaddress = {},
%OPTmonth = {November},
%OPTnote = {},
%OPTannote = {}
%}
\title{A practical aproach to real-time kernels}
\author{Michael Ambrus} 
\date{Nov 2006}
\maketitle


\chapter{Foreword}
\label{FIXME}
	TinKer is a real-time kernel for very small targets. The work behind TinKer is a summary of ideas that has been in my head for more then 15 years. I finally desided to write them down\ldots

	This paper is intended for an novice audience, curious to get an insight and understanding of what that "beast with many names", ticking somewhere in your computer really is.

	Many thanks to my lovely whife \AA sa and to my kids, who's stood by me insipte of the many long night's and incomeless monts during the completion of this work.
\\\\
	If you're interested in the results and in the ongoing project, please visit http://tinker.sourceforge.net

\tableofcontents
\mainmatter
%------------------------------------------------------------------------------
\part{Writing TinKer - the sleepy kernel}
\chapter{Introduction}
A kernel can be written in numerous ways. TinKer's does not in any way claim that this the only way, neither the correct way to do it. Only that this is one way of doing it, based on a certain set of philosophies, ambitions and ideas. One of them those ambitions is to make the kernel as readable and comprehensive as possible.

One of the original intentions with TinKer from the very start, was to write a kernel that would exemplify a text about kernels (this text actually). Therefore you will hopefully find that TinKer is written in such a way that you will be able to follow the code. Kernel theory is a quite difficult subject not normally taught to computer science novices. My ambition is also to make this knowledge available the fresh mind of a student, on so that he'll be able to approach his upcoming carrier with a toolbox of knowledge. thereby making him able to select better solutions for his projects and to identify the really bad ones (most oftenly those based on FUD and who takes advantage of peoples uncertainty and ignorance),

You will notice that this text does not contain many references. This is intentional for two reasons. The first reason is that most of the text here is based on my own insight and therefore there is no natural references. I.e. by doing this I'm also vulnerable for that some of the conclusions are wrong- I'm delighted however to notice that many of the conclusions are verifiable, and should I detect something thats inaccurate I will naturally change those conclusions.  The second reason is a little more difficult to realize. This subject is very difficult. Most oftenly the difficulty lays in the fact that it's difficult to explain. It's like math, you need to come to "insights". The formulas and what-not are just notation (or language). One has to learn the notation to be able to come to insight\ldots.

When it comes to computer science there are many notations and it spawns from computer languages, to graphical descriptive languages and mathematical expression. For the advanced scientist, using any of these (based on the subject of course) this is naturally the best way. But for the novice, this becomes a threshold most oftenly very hard to overcome in a limited amount of time. Scientific work is normally written in such notation however.

\begin{table}[!hbp]
\begin{tabular}{|r|rl|}
\hline
CPU architecture 	& Tool-chain 	& execution type\\ \hline
x86 			& $GNU_1$	& Linux \\
x86 			& $GNU_1$	& Cygwin/Win32 \\
x86 			& MSVC		& Win32 \\
x86 			& Borland	& Win32/DOS \\
ARM7			& $GNU_3$	& HW\\
ARM7			& $GNU_2$	& HW\\
Blackfin		& $GNU_2$	& HW\\
PowerPC 		& $GNU_2$	& HW\\
C166 			& Keil		& HW\\
8051 			& IAR 		& HW (old discont.)\\
Z80 			& IAR 		& HW (old discont.)\\ \hline
\end{tabular}

\caption{TinKer ports (Nov 2006)}\label{ports}
\end{table}

\marginpar{%
$GNU_1$ glibc\\ $GNU_2$ newlib / HIXS\\ $GNU_3$ newlib / def
}


%------------------------------------------------------------------------------
%------------------------------------------------------------------------------
%------------------------------------------------------------------------------

\chapter{Quick tour - writing a kernel in 30 minutes}
\section{Genesis}
\label{kernel30}
I assume that you have allready written at least one simple program. What is that when you look upon it really? What \textit{defines} a program?

\begin{itemize}
\item A program is an entity that describes or instructs another entity what to do.
\item Doing so by implying a certain set of rules
\end{itemize}

	\paragraph{Sequence}

		In a computer program this boils down to something very similar to a written piece of paper. This might sound very obvious, but when you think of it. to be able to make any sense of what is written, we have to imply a certain set of rules. Normally you'd expect interpretation to be \textbf{sequential} - i.e. one thing has to come before another in a certain order. Another thing is to define this \textbf{order} of execution in terms of left to right, top to bottom e.t.a.
\\
		A computer program is actually very similar. It has a defined order (normally from top to bottom) and it is normall thought of as sequential flow of execution. Think of it as a puncard, with a lot of holes placed in certain places on a piece of carbon paper (this is actually more than an analogy, it happens to be a historical fact of the very early computer memories).
\\
		We'll keep this analogy in mind for a while, even though it is not strictly true in all aspects.
	\begin{figure}[!hbp]
	\begin{dotpic}
		node [
			shape=record,
			style=filled,
			fillcolor=yellow,
			fontname=Courier,
			nojustify="true",
			fontsize=10.0
		];

		edge [
			color="red",
			fontname=Courier,
			nojustify="true",
			fontsize=10.0
		];

		graph [
			rankdir = "TB",
			fontname=Courier,
			nojustify="true",
			fontsize=10.0
		];

		MAIN [ orientation=73.0, label="{\
			---- START ----         | \
			do something   | \
			<m1> jump if somewhere  | \
			<m2> jump if somewhere   | \
			<m3> do something   | \
			<m4> jump if somewhere   | \
			<m5> do something   | \
			<call_from> call something | \
			<ret_from_call> continue do something | \
			---- END ---- }"];

		PROC [ orientation=73.0, label="{\
			<pi>  ---- PROC START ---- | \
			do something   | \
			do something   | \
			do something   | \      
			<po>  ----PROC END ----}"];


		MAIN:call_from:e -> PROC:pi:e 
		PROC:po:w -> MAIN:ret_from_call:w [label=RET];
	\end{dotpic}
	\caption{Flow of execution.\label{execflow}}	
	\end{figure}

	In the graph above, we've exemplified the analogy but also introduced yet two more common properties in a computer program, that of \textbf{execution control} (jump, if, while, case e.t.a.) and that of \textbf{encapsulation and reuse} (proc, function e.t.a.). These properties are not common in normal written languages but there exists other notations. Musical notes are a good example.
\\
	The latter two properties are really not important for our reasoning, but what is important is the idea of of an execution sequence and it's order. The red arrows are attempting to lead you into this thinking. You should now feel confortable with the expression \textit{"thread of execution"} and an analogy of the directed sequence or flow. \textit{"Thread of execution"} can sometimes be shortened with just the word \textit{"thread"}.

	\paragraph{thread}
	\label{genesis_thread}
		Once one has accepted the analogy of \textit{"thread of execution"} one needs to add yet another fundamental property until we can call it a \textit{computer thread}.

		If \textit{"thread of execution"} symbolised the whole thread of execution in a processor, a program thread is thought of a tiny part of this total thread. Now, it's important to realize that it is a part of that totality, and hence shares that "big" threads inbound notion of sequence and order. It's really simple, just think of it as a sawing thread that you've cut in a couple of pieces.

		Notice how we're speaking of \textit{one} thread of execution. In a single core CPU there can only exist one thread of execution. I.e. if that CPU runs with a kernel or OS, all these executives are just seemingly parallel. Sometimes the word concurrent programing is used to point out this fact. Only multicore CPU's or distributed systems can have \textit{true} parallelism. Even though concurrency is  a more accurate term than parallel, in this text we'll use both of them interchangeably.

		Speaking of which, what is the difference between a thread and a program anyway?
		Answer: there is none (it only appears that way).

		Assume that somewhere there exists four functions called  do\_1, do\_2, do\_3 and do\_4 as in the program in table~\ref{hc1}.

		\begin{table}[!hbp]
		\begin{verbatim}
		#include "dofuncs.h"
		#define FOREVER 1
		int main( int argc, char **argv)
		{ 
        		while( FOREVER )
        		{
		                do_1();
		                do_2();
		                do_3();
		                do_4();
		        }
		}
		\end{verbatim}
		\caption{Hard-coded shedule.\label{hc1}}
		\end{table}

	You will have to take my word for it for now. But those function calls in the above example are very similar (almost identical as a matter of fact) to threads.
	\\\\
	Imagine circles in in the diagrams that follow (figure~\ref{simpl1} and figure~\ref{simpl2}) representing functions, the arrows representing calls \& returns and the adjacent number the execution order.

	%\clearpage
	\begin{figure}[!hbp]
	\begin{dotpic}
		node [shape=circle,fontsize=10.0];
		edge [fontsize=10.0];
		graph [rankdir = "TB",fontsize=10.0];

		do_1
		do_2
		func	
		do_3
		do_4

		func -> do_1		[label="1"]
		do_1 -> func		[label="2"]
		func -> do_2		[label="3"]
		do_2 -> func		[label="4"]
		func -> do_3		[label="5"]
		do_3 -> func		[label="6"]
		func -> do_4		[label="7"]
		do_4 -> func		[label="8"]
	\end{dotpic}
	\caption{Call-diagram of simple program.\label{simpl1}}	
	\end{figure}
	\footnote{The  middle circle in figure~\ref{simpl1} is the loop in our program in table~\ref{hc1}}
	\begin{figure}[!hbp]
	\begin{dotpic}
		node [shape=circle,fontsize=10.0];
		edge [fontsize=10.0];
		graph [rankdir = "TB",fontsize=10.0];

		do_1
		do_2
		func	
		do_3
		do_4

		do_1 -> func		[label="1"]
		func -> do_2		[label="2"]
		do_2 -> func		[label="3"]
		func -> do_3		[label="4"]
		do_3 -> func		[label="5"]
		func -> do_4		[label="6"]
		do_4 -> func		[label="7"]
	\end{dotpic}
	\caption{Call-diagram of "something else".\label{simpl2}}
	\end{figure}
		
	Now, what is the difference between the graph in figure~\ref{simpl1} and that of figure~\ref{simpl2}?.

	Nothing really. The only difference is that the bubble called do\_1 starts the whole sequence, and it calls the middle circle instead of the other way around. \textit{From that point and onward, they are exactly the same.} Yet, this is conceptually speaking really all that differs between a a kernel with it's threads and any other normal program\ldots Imagine if you could have that while-loop, but just start the whole thing from within the first thread (or any of them for that matter).

	The rest of this text aims you convince you that these two graphs \textit{are in fact identical}, and to show how the wish can come true.

\section{The dispather}
	In reality, what differs between the first and second graph is a entity called the \textbf{dispatcher}. A dispatcher does pretty much what the program above does, except that it usually doesn't work with statically linked functions. Instead it uses \textbf{function pointers}. Have a look at the program in table~\ref{hc2}.

	\begin{table}[!hbp]
	\begin{verbatim}
	#include "dofuncs.h"
	#define FOREVER 1
	typedef void *func_poiner();

	func_poiner fu_1;
	func_poiner fu_2;
	func_poiner fu_3;
	func_poiner fu_4;

	void dispatcher()
	{
	        while( FOREVER )
	        {
	                fu_1();
	                fu_2();
	                fu_3();
	                fu_4();
	        }
	}

	int main( int argc, char **argv)
	{
	        fu_1 = do_1;
	        fu_1 = do_2;
	        fu_1 = do_3;
	        fu_1 = do_4;
	}
	\end{verbatim}
	\caption{Soft-core shedule.\label{hc2}}
	\end{table}

	A function pointer is nothing but a variable that keeps the address of a function. Normally with the intention to also call those function, just as if they were any other normal statically linked functions. Many computer languages has similar concepts, but "C" is particularly easy to program concerning function pointers. (BTW, this happens to be what object methods are In C++ - but that is a totally different story).
\\
	The program in table~\ref{hc2} actually defines a very simplified kernel. Only a few conceptual details are missing really. We'll cover them in later chapters. Until then, do please enter~\ref{hc2} in you computer and convince yourself that it really works.

\section{The schedule}
\subsection{A "job list"}
	What would happen if we instead of having separate variables for the functions were to put those in and array like in the picture~\ref{FunTable}?
	\begin{figure}[!hbp]
	\begin{dotpic}
		node [shape=record,fontsize=10.0,style=filled,fillcolor=yellow];
		edge [fontsize=10.0];
		graph [rankdir = "TB",fontsize=10.0];
		FunTable[label="do_1 | do_2 | do_3 | do_4"]

	\end{dotpic}
	\caption{Array of function tables.\label{FunTable}}	
	\end{figure}
	Then just imagine, we could change the order of their execution just by moving them around. Or we could increase one threads frequency by entering it several times in such an array\ldots We could even make the array larger than needed and put some kind of a stopper at the end\footnote{These entities are commonly known as \textit{"sentinels"}}. That we could alter that list whenever needed in run-time\ldots
	\begin{figure}[!hbp]
	\begin{dotpic}
		node [shape=record,fontsize=10.0,style=filled,fillcolor=yellow];
		edge [fontsize=10.0];
		graph [rankdir = "TB",fontsize=10.0];
		FunTable[label="do_3 | do_1 | do_2 | do_1 | do_4 | do_1 | \<NULL\>"]
	\end{dotpic}
	\caption{Reordered array. Notice that do\_1 occurs 3 times.\label{FunTable2}}	
	\end{figure}

	\begin{table}[!hbp]
	\begin{verbatim}
	#include <stdio.h>
	#define FOREVER 1
	#define NUMTHREADS 100
	typedef void *func_poiner();
	func_poiner furray[NUMTHREADS];

	void do_1(){printf("Run 1\n");}
	void do_2(){printf("Run 2\n");}
	void do_3(){printf("Run 3\n");}
	void do_4(){printf("Run 4\n");}

	void dispatch()
	{
	        While (FOREVER)
	                for(i=0; furray(i); i++ );
	}

	int main( int argc, char **argv)
	{
	        furray[0] = do_3;
	        furray[1] = do_1;
	        furray[2] = do_2;
	        furray[3] = do_1;
	        furray[4] = do_4;
	        furray[5] = do_1;

		dispatch();
	}
	\end{verbatim}
	\caption{Scheduled execution.\label{schedued1}}
	\end{table}
	
	If you want, you can try to modify the program and let one of the threads modify the schedule. Notice though that the thread\footnote{Our \textit{"dispatcher"} requires the \textit{"threads"} to be non-closed. I.e. they have to finish in one run} has to be open\footnote{This is still the case even in TinKer, but since we have multiple entry and exit points in TinKer, it's doesn't have to be from \textit{"start"} to \textit{"end"}.}.

\subsection{Some \textit{jobs} are more important than others}
A very common issue in control systems is the need to make a difference between the importance of certain activities. For example, interacting with the user can be considered less important than acting on a certain certain event. The less important activity can then be postponed or held back for a while until the important job has been completed.
	
	\begin{figure}[!hbp]
	\begin{dotpic}
		graph [rankdir = "LR",fontsize=10.0];		
		style=filled;
		color=white;
		node [style=filled, color=white, fontcolor=white];
		N0 -> N1 [label="priority"]
	\end{dotpic}
\\
	\begin{dotpic}
		graph [rankdir = "TB",fontsize=10.0];
		subgraph cluster_0 {
			style=filled;
			color=white;
			node [style=filled, color=white, fontcolor=white];
			N0 -> N1 [label="order"]
		}
		node [shape=record,fontsize=10.0,style=filled,fillcolor=yellow];
		edge [fontsize=10.0];
		graph [rankdir = "TB",fontsize=10.0];
		FunTable[label="\
			{ T3    |   T1  |   T7  | \<0\> |       |       } | \
			{ T2    |  T5   | \<0\> |       |       |       } | \
			{ T4    |  T6   |       | T8    | T9    | \<0\> } \
		"]
	
	\end{dotpic}
	\caption{"A two dimensional schedule. The 2:nd dimesion is our \textit{"priority"}.\label{FunTable3}}	
	\end{figure}


\begin{table}[!hbp]
\begin{tabular}{|l|c|c|c|c|c|}
\hline
		& Mon 	& Tue	& Wed	& Thu	& Fri\\ \hline
 8:15 -  9:00 	& math	&	& Phys	& prog	& play\\
 9:00 -  9:45 	& math	& music	& Phys	& prog	& play\\
10:00 - 10:15 	& hist	& wood	& math	& comp.sc	& electronics\\
10:15 - 11:30 	& hist	& wood	& math	& comp.sc	& electronics\\ \hline
11:30 - 12:15 	& LUNCH	& LUNCH	& LUNCH	& LUNCH	& LUNCH\\ \hline
12:15 - 13:30 	& Eng	& relig	& biol	& C++	& mechanics\\
13:30 - 14:15 	& Eng	& relig	& bio	& C++	& mechanics\\
14:15 - 15:30 	& art	& music	& chem	& Java	& rocket.sc\\
15:30 - 16:15 	& art	&	& chem	& Java	& rocket.sc\\ \hline
\end{tabular}
\caption{A school schedule inspired TinKer's schedule}\label{school_schedule}
\end{table}


The program in table~\ref{schedued2} shows you a simple example of how to implement and run this prioritised schedule. We're closing in on how Tinker's schedule looks like. 

Tinker's schedule follows the basic outline of our schedule so far very closely. However we can't use function pointers at each cell. If our kernel were to be running threads with a single point of entry and a single point of exit, then we could. But since we want our kernel to be able to do some other nice stuff\footnote{\ldots{}like sending and receiving information between threads using queues, semaphores e.t.a.} we need to put something that holds all the information needed to do that. We need to put each threads \textit{envelope} there instead. We call it the \textit{thread control block} or \textit{TCB}.

Yet one more concept is missing. Not everything about the kernels behaviour can be kept in the schedule. We need to apply certain rules of how to interpret it also. The understanding and implementation of these rules are the job of the dispatcher. For example, the fact that the schedule is now two dimensional and that the second dimension is to be interpreted as the priority is such a \textit{"rule"}. 

How to implement such a rule can in turn be dictated by other rules. For example, dispatch threads at the same priority level in round-robin to give them equal chance to run. Handling of cases to avoid priority inversion could be another rule\footnote{Priority inversion can be avoided by temporary rise the priority of a thread to the same level as the highest level of a thread blocked on it. These advanced dispatching principles has not been addressed in TinKer yet.} 

TinKer has only a few of these rules, and they are hard coded in the logic of the dispatcher. In the literature, one sometimes find expressions like \textit{scheduling techniques} and \textit{scheduling principles} used a little carelessly. I would like to point out that there is a difference between the rules of scheduling and the rules of dispatching. Many times when people talk about scheduling principles, they real mean dispatching principles. Using the correct terminology makes it a easier to know which part of the code in TinKer we're addressing\ldots

As a rule of thumb, everything involved with \underline{reading} the schedule and \textit{acting} based upon that added with additional logic is belongs to the \textit{dispatching domain}. Anything involved with permanently \textit{changing} the schedule belongs to the \textit{scheduling domain}.


	\begin{table}[!hbp]
	\begin{verbatim}
	#include <stdio.h>
	#define FOREVER 1
	#define THREADS_IN_PRIO 10
	#define NUMPRIO 3
	typedef void *func_poiner();
	func_poiner purray[NUMPRIO][THREADS_IN_PRIO];

	void T1(){printf("Run 1\n");}
	void T2(){printf("Run 2\n");}
	void T3(){printf("Run 3\n");}
	void T4(){printf("Run 4\n");}
	void T5(){printf("Run 5\n");}
	void T6(){printf("Run 6\n");}
	void T7(){printf("Run 7\n");}
	void T8(){printf("Run 8\n");}
	void T9(){printf("Run 9\n");}
	void NUKED(){}

	void dispatch()
	{
	        While (FOREVER)
	                for(prio=0;prio<NUMPRIO; prio++)
	                	for(i=0; purray(i); i++ );
	}

	int main( int argc, char **argv)
	{
	        purray[0] = {T3,T2,T7};
	        purray[1] = {T2,T5};
	        purray[2] = {T4,T6,NUKED,T8,T9};

		dispatch();
	}
	\end{verbatim}
	\caption{Prioritised schedule.\label{schedued2}}
	\end{table}
	This concludes the basic ideas that's the foundation of TinKer\ldots

\section{Yielding}
	The operation most associated with dispatching in TinKer is the \textit{yield()} function as shown in table~\ref{yield}. Conceptually yield means \textit{telling the kernel that it might switch} me\footnote{I.e. the current thread} out of context and switch someone else instead if needed.
	\begin{table}[!hbp]
	\begin{verbatim}

	void tk_yield( void ){
	   TK_CLI();   
	   PUSHALL();   
	   TK_STI();

	   _tk_wakeup_timedout_threads();
   
	   TK_CLI();
   
	   //Do not premit interrupts between the following two. Proc statuses 
	   //(i.e. thread statuses) frozen in time.
	   thread_to_run = _tk_next_runable_thread();
	   _tk_context_switch_to_thread(
	      thread_to_run,active_thread
	   );   
	   POPALL();
	   TK_STI();
	}
	\end{verbatim}
	\caption{TinKer yield implementation.\label{yield}}
	\end{table}
	The yield function is called by all TinKer public API as a \textit{"side effect"} right before each function exits. The implementation of yield might differ depending on how the kernel is compiled. This example is showing yield as if compiled default. 

	Yield is what makes concurrency work in a non-preemptive environment. It can be seen a sort of intended interrupt, but one thats initiated by the executing thread\footnote{Which is not how an interrupt works}.

	The analogy is half bad, because two of the interrupts major properties are not there\footnote{Asynchronousity and externl initiative}. But the part that's conceptually equal is that when enter the function\footnote{The ISR is called} and if the system detects that our call (i.e. our imaginary interrupt) is a valid event, a certain thread\footnote{A certain routine will be performed accordingly} will be started. 

	If there is nothing to be done, the program \footnote{I.e. our thread} will just continue normally. And this is also the situation when our imaginary interrupt has completed.

	Using yield in your program is not natural, and normally you wouldn't do that. Remember, yield is executed for you and unless you have really long paths of execution between entry/exit point's you don't have to call it.

	In principle, the same technique could be extended to cove each \textit{"C"} line in your code. This would be a kernel that has all the properties of a preemptive kernel, but without the drawbacks associated with stability and nasty bug crashes. This has to my knowledge never been done, the reason is the ratio between application code kernel which would lead to really poor performance.

	As I already mentioned, yield can look different depending on how the kernel is compiled\footnote{This is handled by conditional compilation directives (\#ifdef)}  the version in table~\ref{yield}. The issue is whether to use exclusive preemption\footnote{Build option '--enable-dispatch=EXCLUSIVE'} or not. If kernel is built that way, there is no need for yielding and the public yield will be replaced by a NULL pointer\footnote{I.e. any calls to yield will be skipped}.

	Instead dispatching is expected to take place at the certain event's affecting our system. I.e. external interrupts or timer timeouts. 

	Please note, that your program can still look the same and you don't need to change anything. In principle you should be able to recompile and run yore application in one \textit{"mode"} or the other without any modifications\footnote{This assumes you're code is correct and doesn't rely on \textit{"side effects"} of any of the \textit{"modes"}}.
\\\\
	Previously I said that TinKer was very much influenced by the simple programs so far in this chapter. Yet, our yield function doesn't look anything like a while loop does it? 

	It's true the function doesn't have a closed loop \textit{inside}, but it's still there. The loop is extended outside, and if you investigate the idle thread\footnote{A TinKer bases application usually has a couple of system threads. Two which are always present are \textit{"idle"} and \textit{"root"}.}, you'll find the loop there instead. See table~\ref{idle}
	\begin{table}[!hbp]
	\begin{verbatim}
	void *_tk_idle( void *foo ){
 
	   while (TRUE){
	      tk_yield();
	   }
	}
	\end{verbatim}
	\caption{TinKer idle thread.\label{idle}}
	\end{table}
	
	It looks like the loop is just extended one level. But do not missinterpret this by thinking context switching can only occur here. This is just where it occurs if all other threads are blocked. All threads are expected to interact with the kernel regularly by using it's API, and by doing this they will also pass entry/exit point's regularly where yielding will also be done.

	I.e. yielding can occur at any levels and any priorities. It doesn't impose any restrictions on the caller, i.e. any thread can call it at any time. Whether \textit{if} any action takes place or not is not for the caller to decide, only \textit{when}\footnote{I.e. at the point where the entry/exit point (or yield) is located.} it might do so.
	

\section{Who's the scheduler?}
\textit{This section is an discussion about run-time vs. compile-time scheduling. It's intended for the advanced reader.}
\\\\
The process of "scheduling" is always a matter for the programmer in way or the other. This can either be in the form as the first examples in this chapter, or more or less aided by the computer system the program will run on. The former is known as hard-coded schedules and the second as soft schedules. 

TinKer is a soft schedule based kernel, which is the case for the majority of OS's and kernels, even among those claiming to be addressing temporal space. 

No matter with type of scheduling technique your computer target provides, human intelligence is always at the top. Scheduling should at the very least be regarded as a combine effort by the programmer and the kernel. I.e. for a soft schedule based kernel, the programmer needs to know which API\footnote{API that will change the schedule are those involved in \textit{creation}, \textit{destruction} and \textit{priority changing} of a thread. A few implicit situations will also change the schedule, like exiting a thread\ldots} will change the schedule.

\marginpar{%
\it{We'll use the terminology hard schedule for hard-coded schedule and soft-schedule for run-time modifiable schedule. This might be the origin of the terms \textit{hard-} and \textit{soft real-time}, which is an unfortunate usage of missleading terminology. This text will avoid using the terms hard/soft real-time.}
}

When the programmer \textit{"schedules"} a system design, he usually would not use the same analogy\footnote{in the form of a "spread sheet" or "school schedule"\textit{}} as TinKer does. He'll probably use some formal methodology that eventually renders in a list of certain tasks and priorities. In some cases also perhaps a additional list of a sequence of changes to these priorities according to certain real world events 

When this program later executes, Tinker's internal schedule will be affected in run-time accordingly and will reflect the programmers design in TinKer schedule form. I.e. Tinker's schedule will be set when the program runs. 

The reason I mention this is that the process of scheduling can also be defined in compile-time. Some scholars in the area of real-time in the temporal domain prefer, that method because even the slightest change in the schedule takes a certain amount of time. In a fully temporal deterministic system you need to have control over \textit{ALL}\footnote{Soft scheduling in run-time might be deemed unacceptable in certain cases.} aspects of time, including those modifying the schedule.

Whether this is good or bad, true or false is a subject of discussion that's been going on for decades.

I do agree with the fact that soft scheduling adds complexity that makes temporal analysis more vulnerable and less trust-worthy. But I would also like to point out the fact that this price for using hard coded scheduling techniques is high and it's just not well suited for the majority systems.\footnote{There are some exceptions. For example control system where extreme safety is required. Extreme in this case, is a justified estimation between cost of lives and cost of system} Don't make sloppy decisions regarding this matter, thinking "better safe than sorry"\footnote{Extreme case where hard-coded technology is chosen} or "let's wing it"\footnote{The other extreme case where soft a soft schedule is used}. If you don't know what your doing please \textit{RUN}\footnote{Both these cases are disastrous if decisions are based on the wrong or factsassumptions.}.
\\\\
TinKer provides a reasonable compromise and can address the needs of the temporal domain because $~{a)}$ the time involved in modifying the schedule is in most cases neglectable and in practice it $~{b)}$ either occur seldomly or in a very distinct patterns\footnote{The master-worker programming model is an example where the schedule will be heavily modified in a deductable way.}. Furthermore you can $~{c)}$ read the code yourelf and deduct the times each schedule modifying operation takes. TinKer is also $~{d)}$ really not that complex, and it's not an impossible or impractical task to determine the execution flows involved with schedule modifications.
\\\\
What you should \textit{never} do if addressing the truly temporal domain though, is relaying on \textit{measurements} and \textit{statistical} distribution. In that case you have left any aspects of real-time long behind you, since you're dependant of that ever so small chance that the unknown will happen. I.e. you're breaking the first principle of real-time - determinism.


%------------------------------------------------------------------------------
%------------------------------------------------------------------------------
%------------------------------------------------------------------------------
\chapter{Fundamentals}

	\section{Context and context switching}
		Before we go any further, we need to mention what is meant by a \textit{context} and what a \textit{context switch} is.
	
		\subsection{Context}
\marginpar{%
\it{Contex $\in$ Environment}
}
			The context you could say is the processors \textit{"environment"} - or more accurately, it's not a property belonging to the processor but to the \textit{executive} (or \textit{execution entity}), which in a system without neither kernel nor OS can be regarded as the same thing. 

			It is the memory, certain states it has achieved and it's registers. The memory in turn can be divided in several part's. From a thread's point of view only one kind of memory matters and that is the part constituting it's stack. For fully fledged OS's more than the stack would belong to each executing environment, it would actually have it's own set of all types of memory.
	
		\subsection{Context switching}
			\textit{Context switching} is a just what the words say: Switching one environment with another. In reality this is a little bit more complicated, but not much. Our context is the stack belonging to the thread. Almost everything we need to \textit{switch} is allready there. We only need to store the CPU's internal registers on top of that stack and we're prepared for the switching. The switching itself is a very simple operation of getting the value of the CPU's stack-pointer (that will point to TOS\footnote{In this text TOS is an acronym for Top Of Stack, i.e. the region of the stack close to or at the top}), store that value away in a safe place. The the new theads stack pointer stored on another safe place and enter that value in the CPU stack-pointer itself.

			If that new stack is properly prepared, the rest will take care of itself. The registers will be restored with the ones previously saved on that stack, and when the execution will also return to the new address stored in the new stack.
			

	\section{Processes vs threads}
		Both processes and threads are execution entities. They are conceptionally very similar, almost identical.\\
		\\
		Not to long ago I heard something funny. The Linux community had just received it's first implementation of a working POSIX 1003.1c compliant kernel (i.e. kernel threads). The whole Linux community was praising this great innovation when some wise guy responded that \textit{"the embedded community has been able to do just that for more than 20 years"}\ldots

		Many times these entities were called something else, like \textit{ultra light processes}, \textit{tasks} e.t.a. In fact they were all really just threads.

		Thread had been possible to program for many UNIX systems for quite some time now. But threads were considered a property a a process (and still are in many peoples mind). Especially among the UNIX community this perception is very strong. This has probably to do with that the only way to achieve threaded applications was to link in a thread library. In most peoples mind this library somehow magically give the programmer threading abilities. For all what it seemed to be worth, those threads shared the process environment, and threads \textit{"belonging"} to one process could not easily communicate with threads \textit{"belonging"} to another.

		There is no real or natural reason for this dependency between threads and processes. Unfortunately this has led to the missconception that this is the case. The only reason why this relation ever existed (because for all practical concerns it really did exist), was that the magic black box in the for of that library that was linked in was nothing else but a mini kernel in itself. I.e. each process would execute this own mini kernel that would in turn run all the threads.

		This is actually also possible with TinKer and some other embedded kernels (some call it collaborate mode - I prefer not to call it anything since there allready exists way to many words and names and concepts). It's just a kernel executing in another kernel, nothing really magical about that. It could be recursed in arbitrary depths if one would want to.

		In reality threads and processes are conceptually equal and dependency between them could either be there or not- That is totally dependant of the implementation. In Linux kernel since 2.5.7 with the addition of kernel threads (which is a missleading name b.t.w. since one would imply they would execute as UID root, which is not accurate) a process can create threads without using an external library. These threads are scheduled in the kernel as if any other process except that context switching between them is much faster.

		Another disadvantage with the older UNIX threading method was that time keeping was really bad. Suppose a thread were to go to sleep but during that time a whole new process would be switched into scope - this poor thread would have no way of regaining execution control no matter how important it would be. It would simply have to sit and wait until the kernel in the OS would let the process that in run the kernel of that thread to regain control. In reality this worked really badly and was almost useless. There would in fact be no particular advantages tread programming than program with processes, except maybe that the API was a little better and easy to handle.



		\subsection{Threads}
			Both processes and threads are execution entities. They are conceptionally very similar, almost identical.

			A thread is just as I explained before, a just piece of that \textit{"sawing thread"}\footnote{The analogy with program threads as cut-off pieces of a sawing thread was first mentioned in section \ref{genesis_thread} on page~\pageref{genesis_thread}.}, i.e. a small piece of the processors complete thread of execution. However, to be usefull it also has it's own context attached to it and the context to a thread is just it's own stack. On this stack, all local variables are stored there during is lifetime. Furthermore, when the thread is sleeping (or blocked as it's more accurately called), the stack will also store a copy of the contents of all registers that the thread had before it was put to sleep. This is so they can be restored when the thread is later awakened.

		\subsection{Processes}
			Processes were to my knowledge actually invented before threads (or at least sort of).
			I believe that during the time the first multitasking execution environment was invented by the guys at AT\&T, it was believed that a process should have all the properties of a CPU. During early days, tools were quite immature and software bugs were not only frequent, they were also very hard to find. Also, since hardware were immensely more expensive at that time, one \textit{HAD} to share the processor with other tasks and stopping the whole system if one process crashed was avoided at all costs. 

			The only way to solve this was to separate each process totally from each other. The context was expanded to contain \textit{all} memory types including the code segment. Now, unless the bug was part of the kernel itself, all other processes could (in theory) continue even if one certain process failed.

			In practice this didn't work out all to well in all cases. For a certain type of systems, for example national database servers or registry accounting systems this would work fairly well, but for embedded systems a failure in one part of the system would very often lead to a total system failure anyway.

			The only way to make truly fault tolerant (and reliable) embedded systems seemed to be to distribute them. This observation in combination with the fact that processes are much more costly to handle (i.e. context switch, start and stop) led to the usage of threads quite early on.\\

			\framebox{%			
			\fbox{Theads and processes are conceptually identical.}
			\fbox{All that differs is that the process context is wider.}
			}\par

	\section{stack}
		Suppose our kernel would be implemented as the simplified kernel in chapter 3. If two threads would be executing the same function, both threads will share that functions local variables. This is normally not what we want. We want that each thread executes in it's own environment with it's own copy of any local variables.

		For that we need one separate stack for each thread. A stack is just a piece of memory somewhere intended to store local variables and function call return addresses.
		\begin{figure}
		\begin{dotpic}
			node [
				shape=record,
				style=filled,
				fillcolor=yellow,
				fontname=Courier,
				nojustify="true",
				fontsize=10.0
			];

			edge [
				color="red",
				fontname=Courier,
				nojustify="true",
				fontsize=10.0
			];

			graph [
				rankdir = "TB",
				fontname=Courier,
				nojustify="true",
				fontsize=10.0
			];

			stack [ orientation=73.0, label="{\
				..  | \
				..  | \
				TOS --\> var 1         | \
				var 2   | \
				..  | \
				..  | \
				var n  | \
				ret address | \
				arg 1   | \
				..  | \
				..  | \
				BOS --\> arg n }"];
		\end{dotpic}
		\caption{A typical stack. Addresses from top to bottom.\label{stk1}}
		\end{figure}

		With our a kernel or OS, a CPU would only have one stack\footnote{Figure~\ref{stk1} shows the principle of a typical stack, with low addresses from top to bottom. In the figure TOS represent top of stack and BOS bottom of stack. When function calls are made, the arguments are first put on stack, then the return address. The function called upon is in turn reserving space for it's local variables on top of the same stack. A stack is said to grow upwards (imagine a hay-stack with more and more straws = data.) The order of the actual elements\footnote{Return address, function arguments, local variables} might vary depending on architecture and calling convention.} which is set during initialisation. When you have a kernel running, each thread (or process for that matter) will execute on its own stack. In principle all one has to do is change the stack pointer between all context switches.

		From a conceptual point of view, the stack concept is unimportant. You could in fact manage without all but the system stack (i.e. the one implicitly inbound by the fact that all CPU uses one), but it would be cumbersome to handle the mentined drawbacks. As a consequence, the stack concept is instead one of the most important entities of a kernel. We \textit{"only"} have to allocate some memory and to somehow attach that memory to each thread. 

		In theory this seems simple, but in reality this not trivial at all. This is also where most people fail, even if they've managed to come to realization as far as to this point. One of the reasons is that to do this operation is not "clean" i programmatical terms. This uncleanliness shows itself among others as need to implement parts in machine code language and towards architecture dependency. However, how dirty this becomes depends on the implementation. TinKer's implementation is to my belief as clean as such an operation can possibly be, and it involves in best case only two lines of architecture dependant code between various targets.

	\section{Time}
		The concept of time exists in many forms in a computer system: real-time, actual time, run-time, compile-time, clock, tick, RTC, process-time\ldots to name but a few. On-top of that, imagine that there exists different "schools" that often interpret one or more of these concepts slightly differently. The difference can be just a minor angel of view, but still have a very deep impact and consequence to the system that's based upon it.

		Time is both one of the best understood entities, and one of the most misused (or least understood) ones.

		To cover all aspects of time in a computer system, their meaning and implications would be daunting task, probably well deserved of a paper of it's own. We're just going to coves some aspect's and what those means to our particular case.

		One major difference worth mentioning is the difference between how we humans perceive time and how computers does is. We humans believe that we exist in a time continuum. Some of us probably also believe that time itself must have started at some point, and that it will stop at some other point. Also, we humans have no real problem to conceptually understand infinitely small differences of time and that two events must be differentiated in time no matter how close to each other they come. One must have been first, and the other must have been second (conceptually we know this even if it sometimes can be hard to see which one\ldots). We humans also often relate to time as absolute.

		All of the mentioned aspects above have a totally different meaning in a copmuter. Absolute time for a computer does not exist without the help of an outer entity telling it so. Just imagine, for a computer all time is relative to one event - then the computer got powered on.

		The other problem for a computer is to implement infinity. A normal computer can't do that since each value has to be stored in some sort of variable thats bound by at least it's representation. It has to trade either resolution or span. What consequence does this lead to? Even though we in any normal calculations get along very well with the value representations we have, time plays it's tricks on us. The reasons are actually quite obvious: 

		\paragraph{Drift} will be a consequence of limited resolution and accumulated error. No matter how small the error is, when infinity is a factor even the error will be infinite. Per definition it's therefore impossible to make any system that is without drift.

		\paragraph{Events} are actually a special aspect of time for a computer system. In many systems one can get away with not having any concept of actual time at all. The order of events could in that case be all we need to know. Some people even argue that time itself is just a continuous steam of separate time events, apart infinitely close to each other. (We'll use this analogy a little bit later in this text, but for now I'll recommend you not to dwell into this). However, one problem even for events also stems from resolution. Since a copmuter has to work with slices of time above infinitely small, there is also the ability (or disability) not to be able to detect which of two event's that occurred before the other.

		In other words: Time is not a very easy concept to make friends with for a computer system.


		\section{Dispatch \& schedule}
		In literature concerning operation systems and kernel technology one often encounters these two words, quite often interchangeably. However they are not the same and in TinKer we've chosen to be picky about these two concepts.

		\paragraph{Schedule:} The schedule in TinKer is a two dimensional array, with priority in one dimension and execution order in the other. Quite similar to a school schedule actually, with days Monday to Friday horizontally and time vertically. In the cross-point of a certain day at a certain time one would get information what one is expected to do. The similarity with TinKer's schedule is striking. In the crosspoint between a certain priority and the ordernumber of that priority one finds thread ID\footnote{A thread ID or TID is a handle to a task control block (TCB). 

		Each thread has one dedicated TCB and this  contains all information the kernel needs to handle the thread. The TID could be seen as an envelope, the TCB as the letter, and the content in that letter as the thread itself.}. Handling of the schedule is normally managed by the kernel itself and is called \textit{scheduling}. Scheduling is more or less planning the schedule, but not executing it. Many things affect the schedule, thread creation and destruction obviously, but also changing a process priority either temporary or permanently affects the schedule. A threads state does however not affect the schedule (i.e. wether a thread is blocked, ready or running), this is a matter of the TCB. 

		Some kernels have their schedules abstracted yet one level in queues, hence the word ready queue, which simply means that all threads determined to be viable for executing are put in a FIFO queue. TinKer does not do this for the following reasons: The FIFO order is allready inbound in the schedule itself since the second dimension is the order to be executed in each priority. All we have to do is letting the dispatcher remember which turn it had executed last time for each priority and then take the next one in turn. The second reason is that of speed and performance. If we were to have a queue, this would work fine if the schedule stays permanent. But what happens if the schedule changes and those changes happen to affect the threads in the ready queue? We would have to re-sort them every time and sorting is a relatively costly operation for a kernel not handling but a handful of threads.

		In big OS:es the situation is different. There you would normally have thousands of threads\footnote{processes actually, but we agreed that these were conceptually equal didn't we?}. Strange at it might sound, sorting is relatively less costly if managing many threads due to optimising algorithms. Also, the schedule itself will change relatively seldom, perhaps a few times per second. Whereas TinKer is made to be able to very effectively run master-worker models\footnote{I.e. implicitly create a lot of small theads with a very short lifetime - this is the operation that by far is the most demanding on the schedule} and for that programming model to be any usefull we have to have a kernel designed to be effective for that.

		As a comparison, I once made a master-worker model implementation sorting a text. It was a silly example, but each word got a dedicated thread whose sole purpose was to move the word it was responsible for into it's correct place. We were litterally starting and stopping several hundreds of threads per second. The same text was then sorted with the same program, one executing under Tinker's control, the otherone under Windows\footnote{with a special POSIX adaption layer, called \textit{pThreads Win32}.}. Guess which one was fastest? Sorting under TinKer took less than 1.5 seconds, whereas the Windows version took more than 27 seconds. Worth noticing is also that the CPU where TinKer was executing was a 40MHz XC167 nd the Windows one was a 3GHz P4 (!).

		In other words, some really beautiful programmatic concepts you can \textit{kiss bye-bye} on a system not well suited for it (which is a shame really).

		\paragraph{Dispatcher:} The dispatcher is a piece of code using the schedule. The schedule is it's "rule set" and the dispatcher dispatches the work based upon it. It does this by determining which thread is supposed to run next and also executing the actual context switch. One says that threads are "dispatched" according to the "schedule", i.e. they are executed one after one in an order determined by the contents of the schedule.
		
		In TinKer additionally keeps track of a order counter belonging to each priority. It does this so it can dispatch work evenly among threads within the same priority. This is needed to avoid the need of a ready queue.

		\section{events vs time}
		events vs time

		\section{Entry and exit points and yield}
		If we were just to start threads, the them execute until they're finished and the die by them selves life would be really easy, and writing a kernel would indeed be possible to do in 30 minutes.

		Even though we \textit{could} write our programs for that type of kernel, we normally wouldn't want to\footnote{At least one very obscure kernel I know of actually expects programs to be written this way.}. The reason is very simple but perhaps not so obvious. It turns out that it's just much more convenient to have threads living longer than just a fraction in time\footnote{Real-time temporal domain analysis using rate.monotonic is especially suited for this type of kernels} and it makes programming much more intuitive. The trade-off urged by the real-time gurus belonging to the temporal school is that you will not have control over which execution path each thread takes.

		This is however easily overcome by programming discipline. If you absolutely must handle temporal real-time requirements in software, you'd be really silly if the thread handling that requirement would be complicated. Besides, theres nothing stopping you using a thread with a simple run-though body, with only one start and one exit point\footnote{I.e. a simple action function without any logic (complexity) or synchronisation}\ldots

		Entry end exit poit's are places in the threads body\footnote{The function each thread operates on we call the threads \textit{body}. Note that several threads can share the same body. The body is only the rules for what to do and nothing else is shered except that. Each thread will have it's own copy of each bodies local variables. POSIX call this the start-function, but I thing body is better because just as a body can have several limbs, a thread can execute in several functions. Each of these functions data will be separated from other threads.} except the the obvious ones (entry and return), where the thread can get in and out of context\footnote{For now think of context as "execution environment" (or just "execution")}. 

		These points are in TinKer most of the function calls you'd use. This is also why the kernel works even on systems unable to preempt. A real time kernel kernel could manage without explicit entry and exit points, but then it would have to base all it's dispatching on preemption. Simulation\footnote{I.e. running you application on another target where the TinKer kernel would execute withing the scope of that targets OS} using such kernels would be very hard if not impossible. Debugging such program would be even worse\footnote{This is based on experience. Tinker is my third kernel, and both predecessors were preemptive-only kernels}. 

		In TinKer we actually have a choice to have preemption enabled only, never have it (i.e. dispatching only inside entry and exit points) or both. Each mode has it's strengths and draw backs, the combination of both potentionally enables us to have the best from both sides.

		\paragraph{Yield} Or yielding\ldots there is a special function intended \textit{only} as an entry/exit point and this one is called \textit{yield()}. It basically does nothing except calls the dispatcher\footnote{In thinker the dispatche is not a stand alone entity and the yield() function is actually part of the dispatcher}. 

		One normally uses the yield() function in parts of the code where you know execution has been going on for long without the kernel having had a chance to try to detect id some other thread is ready to run. Explicit yielding is quite uncommon\footnote{This is due to the fact that most applications will let it's system become idle most of the time. What better to spend that time than to let an dedicated idle thread run the dispatcher over and over again?} and the other API is normally quite enough. Exception to this would be in loops or other programmatical constructs, where the CPU would spend a lot of hard run time.

		

		
%------------------------------------------------------------------------------
%------------------------------------------------------------------------------
%------------------------------------------------------------------------------
\chapter{A thread comes to life}{
A thread comes to life in two stages. The creation of the thread and the staring of the thread. It's important to realize the difference between these two. This chapter will focus mainly on the creation. Start of the thread is covered in the chapter Dispatching in \ref{FIXME}.

\begin{itemize}
	\item Creation of the thread involves allocating and binding it's context. It also involves preparing each threads stack, so that it will be ready for dispatching.
	\item Staring of the thread is done asynchronously by the dispatcher. The dispatche will start the thread whenever it's run and whenever the schedule and priorities are proper for the thread to execute. The dispatcher doesn't know if what it's dispacing is just starting or has allready been executing a while.
\end{itemize}

As allready mentioned, a thread can be viewed just as a sort of a function\footnote{This function(s) we call the threads body. A bodies can be shared among threads.}. Or more precisely a function \textit{call}\footnote{because the thread is really an execution \textit{path} and not the function itself}. The only thing we have to do with it is attach a stack to it and we're ready to go. Or is there something else to it?
\\\\In principle, no - in practice, this is easier said than done!
\\\\The task of designing TinKer can be summarised into only 2 questions\footnote{The solution to each problem starts by knowing how to put the question.}:
\begin{itemize}
	\item Remember the little wile loop in our first program in table~\ref{hc1}\footnote{See page~\pageref{hc1} }? How can we make something that executes the same way, but doesn't start from within the inner loop?
	\item How can we attach a context to a tread\footnote{I.e. how can we connect a thread with it's stack}?
\end{itemize}
It turns out that the solution of both these questions depend on each other. I started by addressing them separately and fortunately I found a solution that satisfied both. I guess I was lucky\footnote{But not really...}\ldots


\section{Creating the thread}
\label{MakeOfThread}
	TinKer actually started as the program in table~\ref{hc1}. For quite some time I was fascinated by the beauty and simplicity by it. My previous kernels were much more complicated in this certain aspect, and I couldn't get rid of the idea that if I just could let that inner loop exist as a function somewhere else than in the main scope, I should be able to dispatch with exactly the same code. Guess what? It was possible!

	\paragraph{Resource allocation} 
		Creating a tread consists of two parts. Allocating the threads resources. This involves getting (or allocating) a TCB\footnote{Task Control Block - the threads \textit{"envelope"}}, allocating a stack and binding the thread, the TCB and it's stack togeather.

	\paragraph{Preparing it's stack} 
		Before we can run the thread we need to prepare it's stack. To realize why this is necessary, one must realize the very tight relation between this part of the threads life, and the dispatching\footnote{We'll cover dispatching in more detail in \ref{FIXME}} of threads. The stack, or actually the upper part of it, \textit{must look like as it would if the thread was swapped out after a context switch}. The reason being that it is the dispatcher that starts the thread and it doesn't do this any differently than any other normal context switching.

		In the preparation of the TOS\footnote{TOS (Top Of Stack) is used as a general term to refer to the whole upper part of a stack, not just the most topmost address. Note that each thread has it's own stack and that we hence have lot's of "TOS".}, three parts can be identified\footnote{It's not necessary to make this separation, but it makes a difference on the implementation since each of them implies certain special ways of solution}:
		\begin{itemize}
			\item Pushing the CPU current registers on TOS\footnote{Technically speaking this is not required, we only need to move the SP the corresponding distance to satisfy the reversed operation of the dispatcher}. We use the same routine that the dispatcher would use for half a context switch. This makes it easier to make certain that the stack looks the same the dispatcher would have left it after a context switch.
			\item Putting the return address to the start of the threads body at the correct place.  Put the threads starting arguments on the stack. \footnote{This step could be viewed as two parts (or steps), but they are so tightly coupled togeather that problems/solutions related to one also implies/affects the other. They belong together\ldots}	
			\item Copy the content of the CPU's current\footnote{I.e. after the above operations. Order is very relevant} SP to the corresponding thread's TCB "SP\footnote{A single-core system can only have one true SP, which is a property of the CPU}" variable. \footnote{This involves only 2-3 lines of code, but needs nevertheless to be written in assembly.}
		\end{itemize}
		These operations are made by a set of macros, why? Two reasons: $^{a)}$ The same reason as any normal program, you want to increase the abstraction level and thereby make it easier to read and logically grasp the code. Normally one would use functions for this\ldots $^{b)}$ Using functions would however be be awkward (if not impossible) since they them selves operate on the stack\footnote{Functions use the CALL operation, i.e. they PUSH, POP and BRET on the stack. Also note that tool-vendors do not lay out the "call stack" exactly alike.}.

		The choice is between hard coding this whole part of the kernel differently for each target, or to use macros as a fair compromise and just hard code those macros differently.

		During the evolution of TinKer the steps above were refined one by one and in innovative jumps. From having to implement all of the steps in hard code, to successively make each of them simpler and simpler. That todays latest solution in practice only needs to address the last point.\footnote{This is experimental as of this writing. But it's seems very promising and if it turns out OK - crude porting of TinKer to a new architecture would be really painless (days, \textit{not} weeks or months) and porting to virtually hundreds of CPU's would be manageable even by a small group of maintainers.}

		
	\subsection{Moving forward, by going backwards}
		To see the problem in the first question, consider the two programs\footnote{These are greatly simplified and would not work in reality. They just serve to exemplify the question.} \ref{hp1} and \ref{hp2} and ask youself what the difference between them is and how we can make \ref{hp2} work. One thing is obviously different. In the series of function calls, the last one is replaced with main\ldots

		If we were to look at the stack\footnote{so far we have only one stack, the system stack} in the middle of execution, and compared them with each other, we would find out that they are surprisingly alike\footnote{As a matter of fact they are conceptionally identical}. If we could just somehow get our program counter inside the loop in \ref{hp2}, the execution flow and the stack comparison would be "exactly" the same. However, our programming language doesn't seem to allow us to jump right inside. Actually, here is the first obstacle\footnote{You can't use C everywhere if you want to design a kernel.}, we need some sort of trick\footnote{But you should try very hard to use C as much as possible}. It seems obvious that we need to use assembly language to solve this problem. But should be just jump? And where to BTW, we don't seem to have a liable to jump to and surely we can't be expected to edit the output of the compiler just to add a liable each time we want to start a new thread?

	\begin{table}[!hbp]
	\begin{verbatim}
	#include "dofuncs.h"
	#define FOREVER 1
	int main( int argc, char **argv)
	{ 
	       while( FOREVER )
	       {
	                do_1();
	                do_2();
	                do_3();
	                do_4();
	        }
	}
	\end{verbatim}
	\caption{Hard-core schedule.\label{hp1}}
	\end{table}

	\begin{table}[!hbp]
	\begin{verbatim}
	#include "dofuncs.h"
	#define FOREVER 1
	int dispatcher()
	{ 
	       while( FOREVER )
	       {
	                do_1();
	                do_2();
	                do_3();
	                main();
	        }
	}

	int main( int argc, char **argv)
	{
	       while( FOREVER )
	       {
	           yield();
	       }
	}
	\end{verbatim}
	\caption{Crude dispatcher and yield.\label{hp2}}
	\end{table}

		No, we don't want to mess with that\ldots The answer lays in how the compiler executes a call\footnote{Hmm, lets see\ldots what was it that's different again?}. 

		Simplified a call to a function is done by a \textit{CALL}\footnote{some CPU architectures don't have any \textit{CALL} in their reportoar. The compiler will in that case construct a series of operations with the same effect based on for example \textit{JUMP}} instruction. This in turn leads to that the next address to be executed \textit{after the function is complete} is pushed on top on the stack. The CPU then jumps to the function and starts execution. 

		When the function is complete, the CPU executes a RET, which just fetches the address previously PUSHED by the CPU and enters that address into the program counter\footnote{The end result of completing this instruction cycle is that the CPU continues execution beginning at the address now in PC}-

		Howabout if we reserve this order in which a \textit{CALL} is executed? Let's try mimicking how a compiler would lay out the instructions . \textit{BUT} the other way around! Lets call the function by pushing it's address on top of our stack and then execute a RET instead. Now why on earth would we want to do that? 		

		Because $^{a)}$ we actually don't need to execute that RET\footnote{This will be done automatically when we exit the scope of the \textit{current} function} explicitly, and $^{b)}$ we can postpone the start of execution by letting the dispatcher \textit{start} the tread for us. The dispatchers job is just too seek up theads to run and then to change the context from the running one to the one to be run. 

		The beauty of it, is that \textit{the dispatcher doesn't know} that it's actually also \textit{staring} execution of threads now and then. All it does is enter a function called yield() and when that function \textbf{\textit{}RET}urns, the new thread will start executing. 

		This way a thread can be made to obey priority from the very beginning. I.e. if a thread is created with a priority less than the one of it's creator, it will \textit{not} start until the creator is ready to become blocked, thereby we don't create priority inversions\footnote{An unexpected bonus is that it virtually doesn't take any time at all to create threads. Time for creation and destruction doesn't have to affect the creator of the thread, who most oftenly is also a thread. Nice, we can now implement master-worker model based applications and don't need to rule this beautiful programming technique out because of poor performance!}. 

		In principle, all we need to do is pushing that address and the rest will take care of it self. Oh and yes, before we do that we also need to change the stack...

	\subsection{We need our locals}
		The second major problem was to attach a context for each thread. This is honestly not very hard at all, but we need another trick. The context for each thread is expected to be reached by \textit{one}\footnote{and one only \ldots} variable. This happens to be prepared by most CPU's allready and is called the CPU's SP\footnote{I.e. the CPU's \textit{stack-pointer}. This is a special register in the CPU that constantly points at TOS (Top Of Stack).}. So our context is pointed out by this SP?

		Not quite, but almost. Each thread has a TCB\footnote{Task Control Block, see \ref{FIXME} which is struct holding several variables. The most important of these variables are each threads SP\footnote{Actually, a copy of each threads SP as it were just before it got blocked. The real SP belongs to the CPU and there can only be one\ldots}} and this is the \textit{key} to the threads context.

		So what we do is\ldots 
		\begin{itemize}
			\item We allocate some memory\footnote{We use TinKer's internal \textit{'kalloc()'} for this, which is a managed \textit{'malloc()'} designed to be thread-safe and execution-time deductable. See later chapters regarding problems with stdlib and multithreading \ref{FIXME}}.
			\item We take the address of the last byte in that chunk of memory, and put this value into the threads SP \textit{copy} in it's TCB.
		\end{itemize}
		That's really all! Now when the dispatcher runs, it will find the thread to run\footnote{which in this case happens to be a newly created thread that wants to enter it's body}. it will look up that thread's context from the threads TCB and it will simply enter that value in the CPU's real SP. The top of that thread has allready been prepared\footnote{because we changed the stack to this one for a moment when we created the thread.} with space for all thee local variables reserved, but most importantly \textit{with the return address} and the threads start-up arguments. The thread with it's stack is now ready to enter it's body for the first time.

	\subsection{Why not facing front}
		Our thread staring technique\footnote{See \ref{MakeOfThread}} so far is elegant. It's very fast and it's quite\footnote{Well, sort of\ldots Very few but lines, but pure murder to implement} easy to implement. But is has a few annoying drawbacks, that all stems from the common denominator that the preparation of each stack before a thread can be run. It involves only a few lines\footnote{Typically between 50 and 250 lines of assembly}, but lines very hard to write. The reason was that besides having to learn each architectures assembly language., I also had to cope with different variants depending on the toolchain. And I had to learn each combination of tool-vendor and architecture there particular methid of calling convention\footnote{Once upon I time I thought there could be only one way. I soon learned that there are many\ldots (similar most of them, but still not quite equal)}. Added on top are the following differences concerning preparation of the top of the stack:
		\begin{itemize}
			\item They differ between architectures
			\item They differ between tools-chains (even for the same architecture)
			\item Calling convention might differ even for the same tool chain depending of how the tool-chain was compiled\footnote{I.e. which ABI it was configured with (ABI means Application Binary Interface. COFF (common object format file) is one example, ELF (extended linux format) another)}
		\end{itemize}
		For a long time I thought I could not possibly make the abstraction of a thread creation and dispatch any higher, and that this dependencies and obstacles would have to remain. I had to spend a few weeks to a 1-2 months for each architecture port, but it was still much better than my previous two attempts. 200 hundred lines of assembly code, how hard could it be\ldots

		The problem was not how hard it was. I could live with the fact that it took a while to create each crude port \textit{once}, but the maintenance issue was scary. I had by then TinKer already running on 6 different targets, and the time I ended up spending tuning and adjusting these lines soon became predominant. Also TinKer was very vulnerable for any small change the tool vendor came up with and I had to re-validate TinKer for each version of each tool-chain.

		There had to be a better way\ldots

		So I begun experimenting with the thread creation again. It turned out that I could let the compiler and CPU in combination set most of the stack up automatically. The solution was not as elegant as before, and I had to trade away a little time upon each thread creation\footnote{The automatic set-up is made by letting the thread execute for a short while. I.e. \textit{"compile time set-up"} was traded for \textit{"run-time set up"} }

		However, the number of difficult lines dropped to $1/5$ of what they where, and since I was spending most of my time in this jungle, I suddenly got much more time left for doing the fun parts.

		\paragraph{This is how the new technique works\ldots} 

		\paragraph{What about preemption?\ldots} 

	\subsection{Yet some improvements}
	\textit{This section descries an area not yet fully prooven.}\\\\
		During my contributing work for the GNU tool-chain, I had to implement some obscure functions called \textsl{setjump} and \textsl{longjump}. I've seen and read about them before, but they seemed awkward to use. Furthermore, they seemed to focus on error handling and other boring stuff. However, they were part of ANSI so we\footnote{i.e. the group helping out porting the GNU toolchain for \textit{Blackfin}} had to implement them. They were always also written in assembly because what they do is saving and restoring the contents of a CPU registers, which differs between architectures.

		Besides from a tiny little jump in both of the routines, they seemed similar to my \textit{PUSHALL} and \textit{POPALL} macros. Actually, they were conceptually equal\ldots and they are part of ANSI\footnote{I.e. No matter version, tool or vendor, you should \textit{always} have a working set, adapted for the CPU architecture and any imaginable variants of that architecture}\ldots
		\\\\
		HEY!\footnote{If it proofs to work out all right, a crude ports will only differ by two lines of assembly between each other. Suddenly all the ports the GNU tool-chain supports is also potentially availabe for TinKer!}

}%Chapter

%------------------------------------------------------------------------------
%------------------------------------------------------------------------------
%------------------------------------------------------------------------------
\chapter{The kernel scope}
The kernel scope is the adapted environment that makes dispatching, and scheduling possible. It is the outer bounds in whichthe CPU executes.
\\\\
In other words, it's the whole \textit{"program"} as seen by the CPU. 

Remember I said that there is "no difference between a normal program and a threaded program, it only appears that way"? This is part of what the kernel scope is.

The kernel scope would be best understood if viewed from two perspectives: CPU and kernel.

\section{CPU perspective}
	From the CPU perspective the kernel scope encapsulates the tinker kernel and all it's threads. It's \textit{the program itself} and could be compared with a \textit{process}. Except that most target's that TinKer is supposed to run on doesn't have any 

	If anything should be called a process it would be the CPU bare-boned execution environment itself, i.e. the \textit{processor}. 
	\textit{process}.%
	\marginpar{%
	\it{processor = process}
	}%
	However, when you run any TinKer based application under for example Linux, the application will indeed execute in a process of it's own - i.e. a Linux process in that case 

\section{Kernel perspective}
	From the kernel perspective, imagine what the kernel would need to be able to run at all. We say that to be able to execute, we need to instantiate the kernel. This is consists of the following tasks.

	\subsection{Allocation}
		One of the main design philosophies behind TinKer is that the kernel shold not rely, trust or depend on any specific libc\footnote{"C" standard library} implementation. However, we don't want to reimplement everything either and would like to take advantage of whatever support the standard library can give us.
		\\\\
		What's the problem then? The problem is that standard libraries doesn't normally consider any concurrency or multitasking. Furthermore if any particular library do, it's always tightly coupled with a certain kernel. We don't want to reimplement any stdlib, basically because in the case of the client using commercial tools we can't (or we would have to rewrite the whole thing), and in the open source case because we would have to administrate, manage and maintain patches witch is a lot of work. So what we do is \textit{coping} with the standard library you're using.

		The only thing that's in the way of using \textit{any} working standard library, is the matter of concurrency. I.e. if we use the library in a non-concurrent way we're OK. This is what TinKer does. \textit{TinKer \underline{preallocates} any resources it needs, and all the resources the application is expected to need\footnote{This is the reason why TinKer needs all those configuration parameters when you build it.} before execution of the application is started. These resources are then handled internally by TinKer's own API, which takes care of any concurrency issues.}

	\subsection{Set up}
		When all the resources are allocated, we need to set the kernel up. This involves clearing the schedule, initialising all TCB's in the pool and to create the idle thread.\marginpar{\it{main resource = stack = memory}}

		Now the current thread of execution (i.e. the CPU total thread of execution) is \textit{"re-assigned"} a context making it appear like any other thread that can be part of our schedule\footnote{Now two threads are in the schedule, idle and root}. This "thread" will be named specially and called "root" and will be given the system highest priority\footnote{Thereby guaranteeing execution until the user program starts it's own first thread}. It's resources will also be based on what was available in the kernel scope, and not allocated as for the rest of the threads\footnote{I.e. it inherits the stack from the kernel scope}.

		Last of all the root's body is entered. This function is expected to be named \textit{'int root()'} by the linker, but is converted by a macro trick to look like any normal program entry point, i.e. \textit{'int main(int argc, char **argv)'}. 

		I.e. you're program can use either of the two, but it must have one. I would recommend to use the second notation because it makes your program directly portable to any other POSIX environment.




%------------------------------------------------------------------------------
%------------------------------------------------------------------------------
%------------------------------------------------------------------------------
\chapter{Dispaching}

\begin{figure}[!hbp]
\begin{dotpic}
	node [
		shape=record,
		style=filled,
		fillcolor=yellow,
		fontname=Helvetica,
		nojustify="true",
		fontsize=10.0
	];

	graph [
		rankdir = "LR",
		fontname=Helvetica,
		nojustify="true",
		fontsize=10.0
	];

	edge [
		color=black,
		fontname=Helvetica,
		nojustify="true",
		fontsize=10.0
	];

	Dispatcher -> IO		[label="You're next"];
	Dispatcher -> Console		[label="You're next"];
	Dispatcher -> Application 	[label="You're next"];

	IO -> Dispatcher		[label="Yield me"];
	Console -> Dispatcher		[label="Yield me"];
	Application -> Dispatcher	[label="Yield me"];
\end{dotpic}
\caption{Dispatching inprogress\ldots\label{disp1}}	
\end{figure}

\section{Where, what, who?}
Dispatching is the activity where reading \& interpreting the schedule and context switching takes place.

So far we're talked about \textit{"dispatching"} and \textit{"the dispatcher"} as if there were a certain entity in the kernel for that. It doesn't\footnote{At least not one dedicated entity}\ldots

Certainly a kernel \textit{could} have a dedicated dispatcher, and I'm certain that some kernels do. It outlines a very nice abstraction and it could be implemented more or less exactly as our very first imaginary kernel in in table~\ref{hc2}.

However, though this seems nice in principle, there are a some penalties to be payed if implemented that way.
\begin{itemize}
	\item $^{A)}$ The dispatcher would have to have it's own context
	\item $^{B)}$ It would have to run at well defined occasions
	\item $^{C)}$ It would have to run at highest priority
\end{itemize}
Alltogeather, this creates a rather clumsy runtime behaviour. Unnecessary time would be spend running the dispatcher when it's very probably not needed. It would also \textit{require} preemptive run-time behaviour, thereby making it much more difficult to run the kernel in a simulated environment.

When the process\footnote{meant in it's literal sentence, i.e. not as a operating system process} of dispatchting takes place can be divided in the following three philosophies\footnote{This is one of the most important foundations for any  kernel and will have one of the greatest impacts on the design}.
\begin{itemize}
	\item On a regular basis, often driven by the system \textit{"tick"}footnote{Usually as part of the timer interrupt which otherwise only increases a counter variable called \textit{systimer}}. I.e. time is regarded as an series of events with a certain granuality{A term used in computer science meaning how "fine grained" time can be perceived as}, and the dispatcher is driven by those events.
	\item Similar to the above, but more generalised meaning that every outside stimuli (i.e. interrupts) will run the dispatcher. This is the case in traditional real time kernels. Note that time could, but doesn't have to, be regarded this way. When you have the abilities to run the dispatcher on other stimuli, you don't need the time to help you cut the CPU's total thread of execution in pieces any more. Wheather it's done or not is implementation specific for your kernel in question.
	\item The application (i.e. the threads themselves) besides when and where. This is done by explicit calls to a certain $yield()$ function, or by entering certain entry/exit functions\footnote{Usually all of the kernels public API is a potentila entry/exit point.}  This is called \textit{explicit yielding} dispatching.
\end{itemize}

Which of these philosophies to use depends on what type of kernel you want. All of them can potentially satisfy the first part of the real-time paradigm. But not all of them can satisfy requirements in temporal space. TinKer uses all of these philosophies, but with an emphasise on the last one. The main difference between the first two and the last is that the formers leads to preemptive behaviour and the latter does not. 

Many kernels follow one of the above philosophies quite strictly. The reason being that each method affect the design very much deep down in the kernel. Their respective solutions don't have much in common either\ldots. However, one doesn't \textit{have} to rule one on behalf of another\textit{theoretically speaking}. Question is rather if it's practically possible \& justifiable or not. 

\textit{Explicit yielding} is the fundamental type that formed the design of TinKer. It turns out that going from this type of kernel dispatching towards any of the other two is much easier than the other way around. One will end up with a quite lot more code than if any other would had been addressed specifically and exclusively from the start, but the code is quite readable\footnote{Going the opposite direction would be a complete mess to say the least.} and trading between readability and performance always goes in favour to readability\footnote{Many will not agree, but use common sense and ask youself: How many obscure OS:es or kernels out there do you know that's survived more than 2 decades?}. Focusing on readability leads to maintainability, and if some speed or effectiveness (if any) is lost that way, so be it. Computer hardware has a tendency\footnote{This interesting \textit{"fact"} has been true for more or less as long as computers has existed} of doubling it's performance by 100\%  every 18'th month but a kernels dispatching method will stay static for it's entire life span.

The design of TinKer is influenced by the  empirical observation, that of that most of the time a computer system runs, it does nothing but sits and waits\footnote{I.e. executes in the \textit{idle loop}}. I figured, as well ass sitting there in a empty while loop doing nothing, it's might just as well do something, even if that most of the time turns out to be equally unnecessary. Why not \textit{"dispatch"} there instead over and over.\footnote{In the idle loop? You must be crazy - this can't work!\ldots}

If this were the only place dispatching would take place, this would obviously not work very well. Fortunately this is not the case either. Dispatching takes place on many places\footnote{Depending on how the kernel is configured: Preemptive only, non-preemptive or combined}, most notably is at the \textit{entry of each exit-point}. I.e. every time you code is making a system call that belongs to TinKer, the dispatcher is run. This is what makes concurrency in non-preemptive environments.

All these calls have one common dominator, and that's the yield function\footnote{more or less embedded deeply in each function}. Yield is where the actual dispatching takes place but you still can't call it the dispatcher because it belongs to no-one and anyone can call it. I.e. dispatching is a property of the whole combined system \footnote{I.e. the kernel and it's threads}. If some entity would be pointed out as a dispatcher, it would have be the whole kernel. Notice the two main properties of the dispatcher
\begin{itemize}
	\item Determining who's next to run and run it\footnote{i.e. context switch to the thread in question}.
	\item Repeat this in a never ending loop
\end{itemize}
The only practical difference between TinKer and the small little example in table~\ref{hc2} is that this loop is extended. It's still never ending though\ldots

\section{The sleepwalker}
This method of achieving concurrency seems really silly doesn't it? TinKer claims to have real-time abilities and very fast response times, so how can this be even remotely possible?

First of all, running yield function is not as costly as one would think. Naturally doing it to often would put an unhealthy ratio between CPU time spent in the application and in the kernel, but you won't in most cases.

To realize this you have to realize the difference between the execution path and the program code. The distance between two entry/exit points might seem close in your program code, but in reality they are relatively speaking usually very far apart.

The second thing that motivates this type of kernel is based on the same sleep observation as mentioned before. It is very likely that whenever something happens\footnote{I.e. when an event happens per definition} that the CPU will be executing in the idle thread happily executing yields all the time. I.e. the probable response time will be lightning fast since we don't have to wait for the dispatcher to specifically be run.

\textit{\textit{"Probability"} based execution! But what about that \textit{"real-time"}?} Here's where the kicker line comes in: Speed and response time hasn't anything to do with real-time in a system not having temporal requirements. Only event's matter and the order between them. This is quite likely the only think you need\ldots

There are a couple of really big advantages using non-preemptive dispatching.
\begin{itemize}
	\item It's much easier to debug (compare preemptive debugging as trying to debug interrupt functions). It's very hard and human error and further conclusive misstakes are eminent.
	\item A preemptive kernel never 100\% safe. Reason is that it's very hard to make a good compromise between where interrupts might occur, and where they must not. Playing it safe but by disabling too much would lead to poor behaviour, opening up could mean you miss a spot that's critical. This are is furthermore complicated by the fact that while inside the kernel, the kernel can't use it's normal protecting mechanisms. It's has to rely on HW disabling of interrupts and this is awkward\footnote{The awkwardness is due to that if you want to open up preemption in some sort of complex part of the kernel, you almost always need to know the state of the interrupt disable flag. However, determining this will take yet a few cycles during which time another interrupt might have occurred. How do you know which? Some HW have support for atomic operations of read-modify-write settings of flags however.}.
	\item From practical experience, I know that being able to run a control application in a simulated environment is a big advantage. Depending of the application the possibility to do this varies of course, but it's usually much more common that you'd think\footnote{Consider if you application has parts that could be run in simulated environment even if other parts can not}. The ability to run in a simulated environment is usually very rewarding and your investment in terms of adaption and set-up time is normally payed back with enormous profit.
\end{itemize}

\section{Have it your way then}
However, should you have temporal requirements - TinKer can satisfy these too. You would have to start using interrupts however. If you feel confident with that, then there is nothing stopping you.

When you start use TinKer in a preemptive way, you will be able to address temporal space. But also please note that the response times will resemble those of a normal kernel much more. No more light speed response times with optimistic success rate. Instead you'll get a little slower but deterministic temporal behaviour.

When TinKer is compiled without any options regarding preemptiveness, preemptiveness is actually enabled. Whether you use it or not depends if you use ISR's that call TinKer kernel API.

However, if you plan to use TinKer for temporal requirement I would suggest you compile it with with preemptiveness exclusive\footnote{Build option --enable-dispatch=EXCLUSIVE}. The reason is that you don't benefit from \textit{using} both preemptiveness and non-preemptiveness behaviour at the same time.

What will happen when the kernel is built that way, is that several of the kernel API that were previously also entry/exit functions will abandon that duty and do \textit{only} what their definition says. This leads to that execution time for each thread from event to response is much easier to deduct (and that's what you need to formally validate your temporal requirements).

The time keeping function will also change to the \textit{PTIMER} module instead of the \textit{sleepy timer}\footnote{\textit{PTIMER} is a preemptive timing technique, whereas \textit{"sleepy timer"} is poll based}. Last but not least, it will lead to that the idle loop will not execute yields anymore, thereby making sure that the time spent in the kernel by the CPU also become minimal and deductable. 

I.e. the CPU will execute inside the kernel only when you tell it to, and time for each execution path can be determined and entered in you rate monotonic analysis.
\\\\
I would urge you to carefully consider if this is for you. True temporal domain problems are something for experts.

Yet another aspect is the following, which is the same reason why true temporal requirement applications are quite uncommon in the UNIX world\footnote{Relatively speaking but even among control applications}. Temporal space is a matter of the whole system. The more complicated the system is, the harder it is to verify. Do you have control of all parts? Are all threads in you're control? Do you even know about them? Can you trust the execution environment?

I.e. even if your timely analysis is OK and you can formally prove your temporal requirements, it's still possible you've missed something that will crash your timeliness, and furthemore: you'll not know it unless it's happening\footnote{I.e. too late}.

So unless you're not 100\% certain you have full control of each part of your system, I would urge you to rethink.

I'm not joking, but the 30 minute kernel might not be such a bad alternative afterall\footnote{See \ref{kernel30}}.

\section{Preemption, events and other stuff}
\marginpar{$Preemption \ne ISR$} Sometimes people confuse preemption with event driven programming, by thinking events are the same as interrupts and interrupts always mean the same as preemption. This is not the case. In fact as table~\ref{pree_ev} will show you, you can use event driven programming technique in a non preemptive kernel, and you can have pollong driven programming while still benefitting from preemption. \\\\
In TinKer most programming techniques work either you use preemption or not and that you don't need to have preemption just to detect event's. Polling used the right way is in many situations both better safer and more predictable. You can also still make use of interrupts even in non-preemptive mode. In that case the thread affected by the even in question will not start executing until the ISR is exited and somewhere else in the normal program an entry/exit point is reached.

The difference is not in the programming techniques, and programs could in fact look exactly the same for the two cases. The difference is in the temporal behaviour of the system. 

Consider the two programs in table~\ref{isr} and table~\ref{poll}. They each represent event  two different ways of interacting with the outside world.

\begin{table}[!hbp]
\begin{verbatim}
unsigned long cbuff[4];
void ASC0_viRx(void) interrupt ASC0_RINT
{
   cbuff[0] = ASC0_uwGetData();
   q_send(tk_sys_queues[Q_SERIAL_0_I],cbuff);
}

unsigned int thread2(void *inpar){
   unsigned long msg_buf[4];
   int depth = 0;

   printf("Thread 2 started with prio 5\n");
   while (1){
      q_receive(tk_sys_queues[Q_SERIAL_0_I],WAIT,0,msg_buf);
      depth++;
      printf("Thread2 (%d) received fron stdin: %c\n",depth,msg_buf[0]);
   }
}

unsigned int cpu_hog(void *inpar){
   int depth = 0;
   int k,l,x,y,z;

   printf("CPU \"hogg\ started with prio 7\n");
   while (1){
      for (k=0;k<10;k++)
         for (x=0;x<1000;x++)
            for (y=0;y<1000;y++)
               for (z=0;z<1000;z++)
                  l = (((x+y+1)*k)/z) % 10;
    depth++;
    printf("Hogg completed round: (%d): %d \n",depth,l);
   }
}

void root(void){
   clock_t latency = 0;

   printf("Hello world o f  TinKer targets\n");
   tk_create_thread("T2",    5,thread2,1,0x600);
   tk_create_thread("HOGG",7,cpu_hog,1,0x600);

   printf("Root started\n");
   while (TRUE) {
      latency = tk_msleep(10000);
      printf("Root \"bling\"!\n");
   }

   tk_exit(0);
}
\end{verbatim}
\caption{Events determined by interrupts.\label{isr}}
\end{table}

\begin{table}[!hbp]
\begin{verbatim}
#define PERIODT 100
unsigned long cbuff[4];
void ASC0_pollthread(void)
{
  clock_t latency;
  
   while (TRUE) {
      latency = usleep(PERIODT-latency); 
      if ( ASC0_uwHasNewData() ) {
         cbuff[0] = ASC0_uwGetData();
         q_send(tk_sys_queues[Q_SERIAL_0_I],cbuff);
      }
   }
}

unsigned int thread2(void *inpar){
   unsigned long msg_buf[4];
   int depth = 0;

   printf("Thread 2 started with prio 5\n");
   while (1){
      q_receive(tk_sys_queues[Q_SERIAL_0_I],WAIT,0,msg_buf);
      depth++;
      printf("Thread2 (%d) received fron stdin: %c\n",depth,msg_buf[0]);
   }
}

unsigned int cpu_hog(void *inpar){
   int depth = 0;
   int k,l,x,y,z;

   printf("CPU \"hogg\ started with prio 7\n");
   while (1){
      for (k=0;k<10;k++)
         for (x=0;x<1000;x++)
            for (y=0;y<1000;y++)
               for (z=0;z<1000;z++)
                  l = (((x+y+1)*k)/z) % 10;
    depth++;
    printf("Hogg completed round: (%d): %d \n",depth,l);
   }
}

void root(void){
   clock_t latency = 0;

   printf("Hello world o f  TinKer targets\n");
   tk_create_thread("T2",    5,thread2,1,0x600);
   tk_create_thread("HOGG",7,cpu_hog,1,0x600);

   printf("Root started\n");
   while (TRUE) {
      latency = tk_msleep(10000);
      printf("Root \"bling\"!\n");
   }

   tk_exit(0);
}
\end{verbatim}
\caption{Events determined by polling.\label{poll}}
\end{table}

 % Writing TinKer - the sleepy kernel
\part{Design philosophies \& considerations}
\chapter{Introduction}
This part will elaborate the design philosophies that is the foundation of TinKer.

One could see these a requirements specification, but I'd rather not. Requirement's in a development project is not scientific per definition, because to be able to set a certain requirement up you really must have a certain idea if it's doable or not, how costly it would be to implement and at least a very general idea of how to do it.

These point's are one step ahead of this. Instead of putting up some inflexible requirements, I put up certain ambitions that I believe are of some certain value to the project. Most of them are not easily verifiable, but still certainly have a deep impact to the end result.
\\\\
It's a matter of prioritising and to figure out what you think is really important\ldots

\chapter{Readable \& Maintainable}
The idea was from the beginning to write a kernel in togeather with a thesis. I allready knew it could write a kernel and also in principle how. Readable and maintainable are closely related terms. 

The idea to back it with a text was tho aid me in seeing the difficulties from the readers perspective. Kernel design is a difficult area to teach. The problem is almost always to "communicate" the ideas and not the techniques themselves\footnote{Because kernels do some "dirty" tricks that cant be described in normal programmatical terms}. 

Once you get over a certain threshold, you'll consider this easy as pie. Problem is to get over that threshold\ldots
\\\\
When one decides upon a requirement like \textit{"readability"} one really doesn't know what one's committing for. How can you make something readable that can't be red from top to bottom? And how do you \textit{measure} this readability?
\\\\
Readability is making everything as clear as possible and not to introduce complicating factors if they can be avoided. Sometimes this aspects goes beyond just choosing between two seemingly equal ways and in some cases a choice has to be made between readability and some other technical aspect\footnote{Performance is something often in clash with readability}. In all cases I've chosen readability because of the main reason of doing this work at all, and for the convince that hardawe will catch up much faster than kernels evolve or new kernels come out 

Readability also has several other very nice secondary effect\footnote{But still very very important} in that TinKer is very easy to port, and hopefully many people will help in doing so.

Porting is an aspect of maintainability. In general you can say that the cleared a code gets, the more maintainable does it also become\footnote{up to a certain level}. However, maintainability is also dictated by the broader technology choices you make. In TinKer's case the idea of having a software based schedule is one example. Common to these decisions are that they are so deeply embedded in your design that they are practically irreversible. They do however define some sort of limit about how maintainable a project can become.

Sometimes maintainability can't only be measured in time spent to do for example a certain bug fixes or any other kind of maintenance.

For example, we've decided that TinKer should also be tool-transparent. This requirement drains a lot of time and work because of the time it takes to verify and test all targets based on these different tools. It's painfull process and it would be much easier to support only for example the GNU tool-chain\ldots

On the other hand, there is a certain value in that TinKer can do this at all. Most kernels can't and the indirect result of that is the gap between embedded developers for small targets\footnote{Devices, 1 CPU based} and traditional embedded developers\footnote{Big system, but still embedded. Traffic control e.t.a.}.

To make a project like this readable and maintainable is not an easy task. You cant address this issue like a normal requirement, like when your requirement is broken you just file a bugrapport and "fix it". The issue is seldom isolated, and I had to re-write certain parts of TinKer many times, not because it wasn't working but because I wasn't happy with the readability.

Basically this is how it's been done, a slow and evolving\footnote{I.e. you don't know exactly how do do it. Instead you permute different solutions until you find one that's satisfying} process, involving occasional re-writes and even re-designs.

\chapter{Standardised API}
One of the most important ambitions in TinKer was to base it's API on well established standards.

The reasons for this are many, but I'll mention a few
\begin{itemize}
	\item Much easier to start using
	\item Thought trough by experts 
	\item Literature is full of books, which saves saves me the a lot of work
	\item Non standard kernels are evil	
\end{itemize}
Tinker follows POSIX standards 1003.1c\footnote{pThreads} and 1003.1b{Only part of. I.e. RT queues and semaphores.}. The ambition is to follow \textit{parts}\footnote{Since TinKer is addressing the small embedded system, we can't cover too much. The important thing is to cover the right parts.} of these standards and not to add any propriority API.

This is important, because it means that $^{a)}$ TinKer will be possible to implement for small to very small targets and still have the important API, and $^{a)}$ you as a developper can transfer your projects as to a bigger and more powerful target as you're needs grow.

The freedom of being able for the user to chose a kernel for the right reasons are very near and dear to me. I want users to keep using TinKer because it's good, not because it's a irreversible choice.

\section{POSIX 1003.1c}
POSIX 1003.1c or pThreads is a fairly recent addition to the full\footnote{POSIX is \textit{HUGE}. It covers all from file-systems to shell and terminal behaviour} set of POSIX standards. describe threads in a UN*X operating system. This is a bit awkward, because tinker is not UN*X and certainly not an full operating system.

However, it's a very good standard, and besides of the above unfortunate and a little bit missleading terminologies, it can be very well adapted and suited for a small embedded system.

TinKer implements almost all of POSIX 1003.1c and the ambition is to cover it fully.

When it comes to similarities \& differences between this standard and TinKer one has to bare a few things in mind:
\begin{itemize}
	\item TinKer is \textit{not} an OS, it's a \textit{kernel}. Many vendors targetting the embedded market call their products \textit{"OS"} which is really not accurate.
	\item There exists no processes in TinKer and the minor parts concerning threads between different processes communicating is not applicable for TinKer.
	\item POSIX 1003.1c is a standard describing an API for concurrency, it does not mention real-time. TinKer is a real-time kernel based on this API. I.e. depending on the real-time requirements, you might be limited in your choices of alternative kernels/OS:es from a portability aspect\footnote{The alternative for the end user would be to use a proprietary API which would be even more limiting}. I.e. you have to find another \textit{real-time} kernel following the same standard\footnote{I.e. there exists kernels/systems that follow POSIX 1003.1c, but that are not real-time\ldots} 
\end{itemize}
\section{POSIX 1003.1b}
TinKer implements parts of POSIX 1003.1b, namely queues and semaphores\footnote{\textit{mqueue} and \textit{sem}}.

This standard is older than POSIX 1003.1c, which is quite funny actually. In my point of view it's a good example of the gap between the \textit{"deeply embedded school"} and the \textit{"old embedded school"}, since this standard among other thing mentions real-time, yet not realizing what real-time abilities depend upon.

So, since it's an older standard one would have to assume that there must have existed some implementation for it somewhere. It would be interesting to see how "real-timeish" this implementation would be.

Some traces can also be found in the API itself\footnote{This is not the case for the API in POSIX 1003.1c, which is a much more modern API}, since it suggests there should be a filesystem somewhere, which is just another unfortunate missconception from the UN*X community. There certainly does not have to exist any filesystem to handle for example queues. However, the standard demands it, so TinKer simulates it's existence.
\\\\
There was no reason to follow the same order of events as the historical order. TinKer implemented pThreads first, and then based the POSIX 1003.1b queues and semaphores on top of that.

The reasons were as follows:
\begin{itemize}
	\item Queues and semaphores are just aspects of the same thing, synchronisation
	\item A system needs only one synchronisation primitive, on which all other can be based on.
	\item pThreads has such a primitive, very awkward to use though.
	\item Some systems I want to use as reference, support pThreads but not POSIX 1003.1b. For these I can use this package as an isolated part.\footnote{For example, if you want to try the examples on a Cygwin Win32 target, you would need TinKer's queues and semaphores. Cygwin is a fine piece of code and I use it myself very often, but it still lacks some parts\ldots}
\end{itemize}


\chapter{Small systems without compromise}
As I mentioned there exists a gap between small systems embedded designers and old school embedded designers.

These two sides have been apart for decades and not being able to come closer each other very much probably because they don't speak the same language\footnote{in technical terms} or because of their backgrounds and traditions are different. Developers targeting small embedded control systems very often have a electronics background thats \textit{"evolved"} into computer science, and they usually have certain difficulties seeing the \textit{"major"} benefits of programmatic methodology, various kernels, standards and what-not.

On the other hand, these guys have a common sense when it comes to what a endproduct is and what costumers believe it's worth paying for.

My ambition is to help these two worlds come closer to each other, by providing a good kernel to the small embedded community\footnote{note: Only the targets are small - the community itself is huge!}, but one that enables programmatic terms that is accepted in the more advances computer science community.

Like in for most of the points in this section, the solution to this issue has the sample properties in terms of evolution, and re-write. However, one aspect make it a little bit to handle, and that is to make ones mind up about what \textit{not} to address:
\begin{itemize}
	\item No filesystems or file based device driver concept
	\item No processes $\Rightarrow$ i.e. no memory protection/handling
	\item End-target support only in terms of over-all architectural ports
	\item Not re-inventing standard library more than necessary
\end{itemize}
I.e. I have a certain type of system in mind to begin with, then I'll implement what's needed for that system and not more.
\\\\
To give you an idea of what that system could be like:
\begin{itemize}
	\item A CPU/MPU with 64k\footnote{CISC CPU} to 128k\footnote{RISC CPU} ROM/Flash
	\item RAM consumption \textit{staring} at 8k-16k
	\item An automotive subsystem like an engine control
	\item High processing power
	\item High volume product with few parts
\end{itemize}
As a rule of thumb, RAM is more important than ROM/Flash. A kernel is a major memory hog when it comes to RAM. In Tinker's case, RAM limits the number of either threads, queues or other kernel resources. TinKer does all resource allocation during startup, but detection is  still in run time. Besides, it will not cover the case if you need to further allocate resources in your application, where you could still get a resource starvation problem that you have to validate on a running system. I.e. if you have an chance to design using more RAM, please do. RAM is cheap and having to much is just not possible by definition ;)
\\\\
ROM limitations are more distinct and deductable in this sentence. If your binary fits when you load it, it will run.


\chapter{CPU architecture transparency}
\textit{The holy grail \footnote{How does one make a kernel CPU architecture transparent? Answer: you can't. But you came more or less close\ldots}\ldots}\
\\\\
One original idea was to see how CPU transparent we could make TinKer. This is not an extremely important issue, because all it means is how long time it will take to make a port. It says nothing about either the kernels technical characteristics nor the ports.

However, it plays a indirect role. If we could make the kernel CPU transparent, there would also be a much greater chance that TinKer got widespread. Thereby creating a bigger community, who in turn would aid in indirect testing and debugging.

To decide to make a kernel transparent as a requirement is just ridicules. One simply doesn't know how much this \textit{"beast"} can be abstracted before one tries. However, havingthis in mind from the start, letting you influence every decisionmaking you do, will lead to something that's much more transparent that not considering it at all.
\\\\
Again, this was a slow maturing process, but the end result is pretty promising\ldots 
\\\\
TinKer has all architecture specific code in in-line assembly macros. These assembly lines are extremely hard to write, since these cover not only architectural difference, but also tool-chain differences\footnote{Calling convention, in-line assembly syntax}. However, number of lines has dropped from an average of 300 lines to just a very few\footnote{Latest experimental code contains only 2 lines of architecture specific code!}. Having only a few architecture dependant lines of code opens up the ability to provide many more ports.

In reality TinKer will easily let itself be ported to all embedded targets that the GNU-newlib library has been ported for, which are quite a few.

\chapter{Tool-chain transparency}
First of, what does this mean?
\\\\
This is a terminology that I'll use in this text to cover a certain aspects that a tool-chain might inflict on a kernel. Most of the kernel is written in C language, and one would think that since C is standardised in itself, this would be enough, This is not the case however\ldots

\paragraph{Context switching} involves the operaton where one has to understand exactly how a function call is performed, so that when the CPU returns from a certain contex switching function, it will in fact return using another threads saved context\footnote{And part of that context is also the "new" return address}.

Different compilers execute a call differently and there are no strict overall rules for this. There exists standards\footnote{Called ABI = Application Binary Interface}, but very commonly these are honoured by only one or a very few vendors.

If this vendor is a dominating party, thus would be OK and we would only have to do one special adaption. But since TinKer addresses an area where there still are no dominating vendor, he have to cope with the fact that there might exist many.

Context switching can be done in many ways, but it's actually not the context switch itself thats the issue. The problem is in creating the thread, since we then need to mimic how a function-call will look on the stack, so that when the dispatcher when it is finally doing context switching will have the newly horned threads stack in a way that is compatible to the dispatches normal\footnote{i.e. context switching with threads previously switched out of context by the dispatcher itself} operation.

This particular area has been one of the toughest come up with a good solution for, but I think we now have a fairly good solution.

\paragraph{Header files \& structure} are usually considered part of a standard library. In fact, the which standard library you use could be viewed as part of the tool-chain\footnote{In this context, the tool-chain means C compiler, assembler and linker}. In this text I've chosen to separate them apart because a compiler can theoretically work with different standard libraries\footnote{For the GNU tool-chain use either Glibc or Newlib, depending on if it's a dedicated embedded target or not (newlib)}.

How various tool-chain separate themselves from their respective standard library differs greatly. In some case a clear separation can't be determined\footnote{Keil} and the two are heavily intertwined, in some cases there exist a fair separation\footnote{GNU} and in some cases the separation is total\footnote{CADUL among others}.

Usually a compiler can't get \textit{completely} free from a C library and some small fragments of code are still needed. For example, GNU needs at least a few modules to help it start up the code. It also uses a few macros to handle endianness and floating point operations are. Furthermore, the GNU tool-chain will copy all the header files from it's corresponding library to a certain place meant to be accessible by the tools.

Very often these default headerfiles clash with TinKer's, and we need to use our own. This is handled by a set of nifty macros. However, to work it requires from the user that the search-patch for system headerfiles come AFTER the search path for TinKer's corresponding files. 

This applies for both application and for building the kernel itself. Failing to do this will in best case be detected in compile time. So far I've never seen a case where an application has managed to build if this is not satisfied, but the compilation errors are really nasty and hard to make sense from.

I.e. Wen you build either application or kernel, make sure you have TinKer's headerfiles coming first in the search-path.

\paragraph{Assembly language}
Assembly language is a particularly difficult area, since it for natural resons can't be standardised as much as a high abstraction language like C.

\begin{itemize}
	\item Different CPU's use different instructions. The dissimilarities are big.
	\item Assembler notation can differ. For example GNU tool-chain commonly use the AT\&T notation. A very powerfull assembly notation, but a bit awkward to use.
	\item TinKer has grouped all it's architecture dependant code in in-line assembly \textit{macros}. Notation for in-line assembly differ between C-compilers.
\end{itemize}
This issue is very hard to address simply by writing the code in one way or the other. The only reasonable way to address it in my opinion, it is to keep the number of lines as low as possible. I.e. make use of the C-compiler as much as possible.

This has led to certain compromises in the kernel, where compile-time set-up has been replaced by run-time set-up. Some few $uS$ are lost, but the trade-off is fair.


\chapter{Standard library transparency}
This is an area of great importance, yet subtle enough to be easily missed as an issue at all.

When I first started out writing TinKer I thought I could get away without any standard library usage at all. In the \textit{"deeply embedded school"} there is a certain dislike against the usage of a standard library, more or less justified.

The reason why, is that it is believed that one looses control over the execution by not knowing what happens. Furthermore, there are certain issues concerning concurrency that are absolutely justifiable reasons \textit{not} to use a standart library\footnote{Or at least, not just any}.

The issue is tightly connected to the fact the traditional most embedded tool vendors didn't offer separable libraries and that the ones they did offer had certain flaws that would occasionally turn up as non-acceptable system behaviour.

The above issues are actually absolutely justifiable reasons, and to my knowledge only one other kernel\footnote{RTEMS} address this "by the book". The school-book way to handle this, is to rewrite or (adapt an existing) a standard library so that it satisfies all those time, behaviour and concurrency issues that you might have. A daunting task indeed\ldots

The issue with libc can be viewed from two angles
\begin{itemize}
	\item The TinKer kernel
	\item Your application
\end{itemize}
I've addressed the first point by letting the kernel have only two function calls it depends upon: \textit{clock()} and \textit{malloc()}. Both which can be handled even without the aid of a libc however.

The second angle is different. Except the API covered by POSIX 1003.1c and 1003.1b, the kernel doesn't aid the application in any way. I.e. you have to make sure youself if a certain function is safe to use in a concurrent environment or not. You can make it safe, it's not very hard. But the project domain has to be limited somewhere, and even if each API is relatively easy to certify, there are number of functions are just to be to address. Especially if you recall that we want to be as independent as possible, i.e. by being able to colaberate with more libc than one\ldots
The things to address when protecting your application
\begin{itemize}
	\item Reentrancy and thread safety\footnote{Functions in this category are all that uses file-handles. Most notably: printf, scanf\ldots}. If a function has \textit{"states"} or some other \textit{"side effect"}, wrap a mutex around it. Best thing to do is banning the usage of the function, and make the wrap in another following a certain naming convention thats easy for the programmer to remember. Optionally you can even wrap this new function around nifty macros, making appear as if it were the original function call\footnote{This is how TinKer does it}
	\item Dynamic memory allocation\footnote{alloc, malloc \'et all}. In an small embedded system you should carefully evaluate if you need dynamic memory allocation at all. You probably have limited resources, and even if concurrency is not the problem, starvation might be\footnote{And this is certainly not nice to find out during run-time}. Many alloc implementations also have an indeterministic execution time depending on if the available memory on the heap is fragmented or not. Dynamic memory allocation has states and is inheritly \textit{not thread safe}. For the GNU tool-chain this possible to handle because of the system function call-outs, and TinKer might address this issue differently in the future.
\end{itemize}
By originally having the ambition to not use any library at all, TinKer evolved to have very few dependencies towards one. Actually, TinKer is absolutely dependant of only one standard library function call to be able to dispatch a schedule\footnote{The primary purpose of any kernel}: \textit{clock()}. And even this is wrapped around a macro, so it can be temporarily dealt with\footnote{at least temporary when when you write a new port. A complete port must have this implemented}. However, no port is particularly usefull without being able to schedule real threads. So to complete a port, implementing following API will make life much easier to the programmer.
\begin{itemize}
	\item clock
	\item malloc
	\item printf
\end{itemize}
What about memory allocation then? You said this bas \textit{"a bad thing"}? It is, but only if used in a concurrent situation. TinKer needs memory for various of purposes, and using malloc\footnote{malloc and friends} is a convenient way to get it. However, we don't have to malloc and free all the time if we preallocate during startup. Instead we use preallocated memory\footnote{Which are sometimes indeed allocated by malloc and friends, but during start-up and before dispatching is started (i.e. when the CPU hasn't yet started to execute concurrent threads) } pools that store various kernel elements until they're needed. You only have to configure/initialise the kernel, and it will preallocate those pools during startup. 

This solves both the concurrency issue and the timeliness issue whit the small price of perhaps using slightly more RAM that the system otherwise \textit{on average}\footnote{Note that praxis and a very good rule of thumb regarding \textit{embedded} applications or \textit{real-time} applications, is to always \textit{\textbf{design by worst}} case (i.e. "on average" is a very bad thing to rely on) } would use otherwise.  However it doesn't solve the applications needs. One way to solve this could be to extend the internal service to cover the application also.

So to make wrap this discussion up a bit, issues with standard library can be dealt in several ways and by several instances. We concluded that whence you've built the kernel, it doesn't need any libc and that you application can handle the remaining issues as follows:

\begin{itemize}
	\item Thread safeting, using function wrappers and optionally additional macro wrappers
	\item Protection on system call-outs (GNU/Newlib only)
	\item Don't use libc.
\end{itemize}




\chapter{Real real-time}

\section{Preemption by need-only basis}
\section{Tinker's dual time concepts}

\chapter{Runnable in desktop as mini-kernels}
aiding of debugging and development

\chapter{Well defined limitations}
no filesys, no OS

\chapter{Source code organisation}
TinKer's source code is organised as in in table~\ref{sorce_tree}. The source in this tree is TinKer code only. Any examples, tutorials and preconfigured projects are not part of this tree. The TinKer project might make such trees available to the community, but it will in such case still not be part of the TinKer tree.
\\\\
The following subsections will explain the TinKer tree.
\begin{table}[!hbp]

\begin{verbatim}
tinker-x.y.x/
   tinker/
      src/
         *.c
         *.h
         arch/
            arm/
               *.h
            bfin/
               *.h
            powerpc/
               *.h
            :
            <CPUARCH>/
               *.h
      include/
         *.h
      lib/
         *.a
      bsp/
         gnu_arm/
            *.*
            /lp21xx
               *.*
         gnu_bfin/
            *.*
         gnu_powerpc/
            *.*
         keil_c166/
            *.*
         :
         <vendor>_<CPUARCH>
            *-*
            <BOARD>
               *.*

\end{verbatim}
\caption{TinKer source tree}\label{sorce_tree}
\end{table}
\section{\textit{tinker/src/} vs \textit{tinker/include/}}
To have a [project]/src directory instead of putting everything in the root-directory, is rather obvious and common. It makes distribution easier and it makes handling of several trees in the same CVS server manageable.

However, separating a \textit{[project]/include/} might not be as obvious.
\\\\
This is done for two reasons:
\begin{itemize}
	\item First of all it makes it very easy to install the headerfiles on the build system. Since we know where they are, and that the directory is not contaminated with intermediate build files, we just copy all of them.
	\item A program like a kernel is a fairly big project. As such we need to divide it in pieces like any normal project would. But because of a kernels nature, it has data that has to be visable by all the various parts of the kernel itself (i.e. all the kernel's files). We bind these togeather using headerfiles but we don't want \textit{those} headerfiles to be mistakenly used by the application programmer (thereby risking to corrupt the kernel in a very nasty way). These headerfiles are kept apart from the ones in \textit{tinker/include/} and are part of \textit{tinker/src/}. They are only used wen the kernel is build and not needed by any application.
\end{itemize}
\subsection{\textit{tinker/src/arch/}}
In this branch, \textit{generic} code relevant to various architectures is stored. Don't confuse this with the branch in \textit{tinker/bsp/}, where \textit{specific} code is stored.

\section{\textit{tinker/lib/}}
This is the directory where libtinker.a will be built but it's also an intermediate area. You'll need to link against libtinker.a from this directory in case you build the kernel with a non GNU tool-chain\footnote{When built and installed with a GNU tool-chain, both the build and source tree can be removed after installation.}.

TinKer will be build as a library. But during the process, it does so by building parts in smaller libraries. These libraries are to be considered as intermediate builds, but we need somewhere deductable to store them for the rules in various makefiles to be common and easy to write.

\section{\textit{tinker/bsp/}}
\textit{BSP} is na acronym commonly used as meaning \textit{"board support package"}. I.e. in this directory code for a specific \textit{"boards"} are stored. \textit{"Boards"} in turn not only imply specific peripherals, but also a certain main CPU architecture, tool-vendor specific stuff and in many cases also a specific CPU of each main family. 

The structure with it's adherent directory naming convention underneath \textit{tinker/bsp/} is there to handle those particular differences.
\\\\
The reason why we keep this tree apart from the other branches are as follows:
\begin{itemize}
	\item We need to limit the project somehow. BSP's represent a potentially very big code base since users are expected to make their own adaptions to fit their specific targets (i.e. BOARD).
	\item I.e. the \textit{tinker/bsp/} branch is a good place to draw that line, but still to provide some sort of "template" and interface to make it easy for those adaptions.
	\item This branch is expected to have a more relaxed policy regarding maintainers and contributors.
\end{itemize}

On this whole branch, maintainers are free to implement their port's as they like. They don't have to follow the same coding standard or use any particular files or particular filenames. The only rule is to name the topmost directory following the vendor\_arch convention, and to have one or more sub-directories where the actual code is stored. These subdirectories should be named such that the name implies which "board" it is intended for. If a BSP is (or has) a generic template, one of these sub-directories should be named \textit{"generic/"}.

What binds a BSP togeather with TinKer differs between tool-vendors. For GNU, which is the most common case and prefered choice, this is done by specific function calls that the BSP has to implement. Previously I memtioned the thre mandatory ones, but for a complete port one has to implement a few more\footnote{The functions are defined by GCC}.
 
\chapter{GNU and other thoughts}

 % Design filosofies

\part{Using a pThreads, mqueue based kernel}
%------------------------------------------------------------------------------
%------------------------------------------------------------------------------
%------------------------------------------------------------------------------
\chapter{Introduction}
In this part, we'll start using TinKer by playing with some examples. That the examples are targeting concurrency issues. Real-time issues are a very often tightly related to concurrency\footnote{An aspect of concurrency}. But remember, formally it's not the same.
\\\\
TinKer follows the POSIX 1003.1c \& 1003.1b standards. The first one covers concurrency but real-time is not mentioned at all. In the second standard real-time is mentioned, but the meaning is not complete and addresses queues only.
\\\\
TinKer covers POSIX 1003.1c, but does it so that it grants real-time behaviour to the system that uses it. The end result is a real-time kernel with a standardised, very widespread and very well defined API.
\\\\
All the examples have been verified against two other independent OS's to verify their POSIX 1003.1c and 1003.1b compliance. 
\\\\
The examples will be tailored for a GNU tool-chain. Other environment for TinKer exists\footnote{MSVC, Keil, Borland} however.
\\\\
This part could be extended to cover the subject of concurrent programming and real-time of it's own, which is not the purpose. Literature are full of good books in these subjects. TODO references \ref{FIXME}

%------------------------------------------------------------------------------
%------------------------------------------------------------------------------
%------------------------------------------------------------------------------
\chapter{Building the kernel}
Start a shell and move to the directory where you've the TinKer sources. Now execute the following:
\\\\
\textit{make configure}
\\\\
This is a boot-strap build and will prepare the project to be configured. It has to be run once after each check-out or fresh source package installation. After that, you don't need to reinvoke this command even if you configure and build the kernel for different targets (or any other option changes).
\\\\
Then run the configure script with the appropriate parameters. You can do this either from the source directory root or from an empty directory\footnote{Same principle as most Open Source projects}. For example\footnote{Type  \textit{./configure --help} for a full list of usage and options.}:
\\\\
\textit{./configure -C --host=arm-hixs-elf MCPU=arm7tdmi BOARD=lpc21xx}
\\\\
This will configure the project tor the target architecture ARM\footnote{1'st part of the \textit{host} name triplet defines the architecture, 2'nd defines the system interface and 3'rd defines the ABI}, using the system interface HIXS and calling convention used by ELF. The specific CPU variant arm7tdmi\footnote{See gcc documentation for a list of valid CPU's} is pointed out and the BSP board adaption is lpc21xx.
\\\\
Now all you have to do is:
\\\\
\textit{make all}
\\\\
\textit{make install}
\\\\
This will build the kernel as a library\footnote{libtinker.a} and install this and Tinker's header files on your build system\footnote{Note, to install on your target you still need to statically link with an application and download the binary}.

\paragraph{Removal}
In case you need to remove TinKer from your system, invoke the command: \textit{make cleanall}\\\\
In case you only need to remove intermediate files in your build directory invoke: \textit{make clean}\\\\

\paragraph{When defaults are not enough}
What is built as default \textit{differs} from target to target, depending on what the maintainer of that target felt was reasonable considering it's size and performance. You might want to enter the following configure options to make sure you'll get a build that will work for the examples:
\\\\
\textit{./configure -C --host=\$arch-ven-abi\footnote{For example: \textit{powerpc-eabisim}} MCPU=\$your\_cpu\footnote{For example: \textit{860}} BOARD=\$your\_board\footnote{For example: \textit{generic}} --enable-itc --disable-ptimer --enable-posix\_rt --enable-pthread --enable-kmem=20000 --enable-max\_prio=0x10 --enable-max\_threads=500 --enable-norm\_stack=0x0800}




%------------------------------------------------------------------------------
%------------------------------------------------------------------------------
%------------------------------------------------------------------------------
\chapter{Building your applications}
Whence you have built and installed the kernel, it's now part of your tool-chains installed structure. The kernel is in the form of a library\footnote{libtinker.a} with corresponding headers. You can have several TinKer builds installed on each tool system, one for each part of the canonical name triplet\footnote{Fore example one for each of the targets: arm-hixs-elf, arm-elf, bfin-hixs-elf, powerpc-hixs-eabi \& powerpc-eabisim}.

TinKer gets installed as in the structure in table~\ref{inslall_structure}:

% ./powerpc-eabisim/lib/libtinker.a
% ./lib/gcc/arm-hixs-elf/4.1.1/tinker/crt0.o
% ./powerpc-eabisim/include/tinker/pthread.h


\begin{table}[!hbp]

\begin{verbatim}
PREFIX/
    arch-ven-abi/
        lib/
           libtinker.a
           tinker/
             flash\_gnu.ld
             ram\_gnu.ld
       include/
          tinker/
             *.h
    lib/
       gcc/
          arch-ven-abi/
             x.y.z/
                tinker/
                   crt0.o
\end{verbatim}
\caption{Where TinKer files get installed}\label{inslall_structure}
\end{table}

Due to TinKer's ambition to cope with different libc, it also does not impose any changes or requirements on the headerfiles. Note that libc usually allready has several headerfiles with the same name as TinKer's headers, but with different or incompatible contents.

The TinKer headers are designed to be able to not only cope, but also take advantage of certain parts of any existing headerfiles. It's therefore very important that you application builds headerfile search-path finds the TinKer headers first and the original headerfiles after.

TinKer will aid you in this task by the use of the mandatory define CHAINPATH. Use this define instead of a searchpath to the original headerfiles. It will make it possible for TinKer sources to overlay header files\footnote{I.e. either as whole file substitution or as combined with TinKer's equivalent}. automatically (internally handled by macro expansion).

Just set this define to what the system header-files would otherwise be, and set the \textit{-I \$dir} to TinKer header search path\footnote{See table~\ref{inslall_structure}}.
\\\\
To compile a simple hello.c program, follow the steps\footnote{Put them in a script or makefile} as in table~\ref{build_hello} (please note that the order of each file in the linking part is relevant):
\begin{table}[!hbp]
\begin{verbatim}
export SYSH=/prefix/arch-ven-abi/include
export TINKI=$SYSH/tinker
arch-ven-abi-gcc -c -O0 -g3 -gstabs -DCHAINPATH=$SYSH -I $TINKI hello.c \
   -o hello.o

export SYSL=/prefix/arch-ven-abi/lib
export GCCL=/prefix/lib/gcc/arch-ven-abi/x.y.z
arch-ven-abi-ld -o hello.abi -L$SYSL -L$GCCL  $GCCL/tinker/crt0.o -ltinker \
   hello.o -Ttinker/flash_gnu.ld -lc -lm -lgcc

arch-ven-abi-objcopy -O ihex hello.abi hello.hex
arch-ven-abi-objdump -d hello.abi > hello.objdump
\end{verbatim}
\caption{Build steps: hello.c}\label{build_hello}
\end{table}

I'd recommend that you build and test the examples in this whole part of the text in the following three cases one by one\footnote{I.e. build ALL the examples for each of the mentioned cases one by one. The order is from the easiest to the hardest to get started with.}:

\begin{itemize}
\item As native programs using another OS available to you, for example Linux \footnote{You'll need a Linux kernel 2.6.8 or newer} will work fine. Note that this build does not use TinKer at all and the build steps are also different\footnote{For example: \textit{gcc -O2 -g -o hello hello.c}}. It's for your exercise, to remember to always make certain your applications are portable across targets.
\item Build the kernel for your desktop computer as a target, and link your applications towards this library. This will enable you to run the examples in a simulated environment \footnote{Available simulated targets are Linux x86 and Cygwin (Win32)} which is more convenient than to run it on a dedicated target.
\item Cross build \& run for a dedicated OS-less target and remote debug it. 
\end{itemize}
%
The build process regarding the last point is identical to the simulated version, except that code needs to be downloaded and cross tools need to be available. If you're unconfident about how to set up a GNU cross tool-chain or you don't have a supported target available, the simulated environmen will work just as fine.
\\\\

%------------------------------------------------------------------------------
%------------------------------------------------------------------------------
%------------------------------------------------------------------------------
\chapter{"Hellow world"}
Let's start off by a classical hello world example. The program consists of a main part in table~\ref{hello1} and one thread in table~\ref{hello_thread}.

The initial part in table~\ref{top_of_app} is just a couple for includes and defines for your convenience. Please reuse it for the rest of the examples.

\begin{table}[!hbp]
\begin{verbatim}
#include <stdio.h>
#include <stdlib.h>
#include <pthread.h>
#include <assert.h>
#include <string.h>
#include <errno.h>

#ifndef TRUE
   #define TRUE 1
#endif
#ifndef FALSE
   #define FALSE 0
#endif
#if !defined(TINKER) && defined(_WIN32) &&  defined(_MSC_VER)
   #include <windows.h>
   #define sleep(x) (Sleep(x * 1000))
   #define usleep( x ) (Sleep( (unsigned long)x / 1000ul ) )
#endif
\end{verbatim}
\caption{Top of your source - reuse this in all applications that follow.\label{top_of_app}}
\end{table}

\begin{table}[!hbp]
\begin{verbatim}
void *hello(void *arg){
   int i;

   for (i=0; i<10; i++){
      printf(%s,(char*)arg);
      msleep(1000);
   }
}
\end{verbatim}
\caption{The "Hello" thread.\label{hello_thread}}
\end{table}

\begin{table}[!hbp]
\begin{verbatim}
int main(char argc, char **argv)
{
   pthread_t         threadA;

   printf("Hello Universe\n");
   printf("Root started\n");

   assert (pthread_create(&threadA, NULL, hello, "Hello World!\n" ) == 0);

   assert (pthread_join(threadA, NULL) == 0);

   return 0;
}
\end{verbatim}
\caption{Hello world program (TinKer style).\label{hello1}}
\end{table}
A nice twist to exemplify the difference between a thread and it's body is modifying the program in table~\ref{hello1} to the one in table~\ref{hello2} to let several threads use the same body. Notice that each thread will have it's own copies of all the variables defined as local in that body.

\begin{table}[!hbp]
\begin{verbatim}
int main(char argc, char **argv)
{
   pthread_t         threadA,threadB;

   printf("Hello Universe\n");
   printf("Root started\n");

   assert (pthread_create(&threadA, NULL, hello, "Hello " ) == 0);
   assert (pthread_create(&threadB, NULL, hello, "World!\n" ) == 0);

   assert (pthread_join(threadA, NULL) == 0);
   assert (pthread_join(threadB, NULL) == 0);

   return 0;
}
\end{verbatim}
\caption{Simple modification of Hello World. Two threads use the same body.\label{hello2}}
\end{table}

%------------------------------------------------------------------------------
%------------------------------------------------------------------------------
%------------------------------------------------------------------------------
\chapter{A mutex example}
This example shows the usage of a mutex. Please notice that the mutex variable in this example has to be public to both threads\footnote{I.e. it has to be defined in the code section} and it therefore must be defined before both threads.

\begin{table}[!hbp]
\begin{verbatim}
pthread_mutex_t glob_mux = PTHREAD_MUTEX_INITIALIZER;

void *funcA(void *arg){
   int i,j;

   for (i=0; i<6; i++){
      pthread_mutex_lock(&glob_mux);
      for (j=0;j<20;j++){
         printf("Thread A - In critical section [%d, %d]\n",i,j);
         usleep(200000);
      }
      pthread_mutex_unlock(&glob_mux);
   }
   return arg;
}

void *funcB(void *arg){
   int i,j;

   for (i=0; i<4; i++){
      pthread_mutex_lock(&glob_mux);
      for (j=0;j<20;j++){
         printf("--[%d, %d]--------->>Thread B<< is king!\n",i,j);
         usleep(300000);
      }
      pthread_mutex_unlock(&glob_mux);
   }
   return arg;
}
\end{verbatim}
\caption{Two threads - one critical section.\label{mutex_threads}}
\end{table}

\begin{table}[!hbp]
\begin{verbatim}
int main(char argc, char **argv)
{
   pthread_t         threadA, threadB;   

   assert (pthread_create(&threadA, NULL, funcA, (void*)1 ) == 0);
   assert (pthread_create(&threadB, NULL, funcB, (void*)2 ) == 0);

   assert (pthread_join(threadA, NULL) == 0);
   assert (pthread_join(threadB, NULL) == 0);

   return 0;
}
\end{verbatim}
\caption{Mutex program (TinKer style).\label{mutex1}}
\end{table}
%------------------------------------------------------------------------------
%------------------------------------------------------------------------------
%------------------------------------------------------------------------------
\chapter{First look at queues}

\begin{table}[!hbp]
\begin{verbatim}
void *thread1(void *inpar ){ 
   int         loop_cntr            = 0; 
   int         loop_cntr2           = 0; 
   mqd_t       q; 
   int         rc; 
   char        msg_buf[16]; 

   q = mq_open( QNAME, /*O_NONBLOCK | */O_WRONLY, 0 ,NULL); 
   if (q==(mqd_t)-1){ 
      int myerrno = errno;

      printf("1.Errno = %d\n",myerrno);
      if (myerrno == EACCES)
         printf("EACCES\n");

      perror(strerror(myerrno)); 
      assert("Queue opening for writing faliure" == NULL);
   } 

   while (TRUE) {
      usleep(100000); 
      loop_cntr++; 
      if (loop_cntr>=2){ 
         loop_cntr2++; 
         loop_cntr = 0; 

         msg_buf[0] = loop_cntr2;
         printf("sending....\n"); 
         rc = mq_send(q, msg_buf, MYMSGSIZE, 5); 
         if (rc==(mqd_t)-1){
            int myerrno = errno;

            printf("2.Errno = %d\n",myerrno);
            if (myerrno == EACCES)
               printf("EACCES\n");

            perror(strerror(myerrno));
            assert("Queue writing faliure" == NULL); 
         } 
         printf("sent!\n"); 
      } 
   } 
   return (void*)1; 
} 
\end{verbatim}
\caption{Queue thread 1.\label{q_thread1}}
\end{table}

\begin{table}[!hbp]
\begin{verbatim}
void *thread2(void *inpar){ 
   char        msg_buf[16]; 
   mqd_t       q; 
   int         rc; 

   q = mq_open( QNAME, O_RDONLY, 0 ,NULL); 
   if (q==(mqd_t)-1){ 
      int myerrno = errno;

      printf("3.Errno = %d\n",myerrno);
      if (myerrno == EACCES)
         printf("EACCES\n");
      perror(strerror(myerrno)); 

      assert("Queue opening for reading faliure" == NULL); 
   } 

   while (TRUE) {
      printf("receiving....\n");
      rc = mq_receive(q, msg_buf, MYMSGSIZE, NULL);
      if (rc==(mqd_t)-1){
         int myerrno = errno;

         printf("4.Errno = %d\n",myerrno);
         if (myerrno == EACCES)
            printf("EACCES\n");

         perror(strerror(myerrno)); 
         assert("Queue reading faliure\n" == NULL); 
      } 
      printf("Received: %d of length %d\n",msg_buf[0],rc); 
   } 
   return (void*)1; 
}
\end{verbatim}
\caption{Queue thread 2.\label{q_thread2}}
\end{table}


\begin{table}[!hbp]
\begin{verbatim}
int main(char argc, char **argv)
{ 
   pthread_t T1_Thid,T2_Thid; 
   mqd_t q2;
   int loop =0;   
   struct mq_attr qattr;

   printf("Unlinking old queue name (if used). \n");  
   mq_unlink(QNAME);  //Don't assert - "failiure" is normal here
   sleep(1);

   qattr.mq_maxmsg = 3;
   qattr.mq_msgsize = MYMSGSIZE;
   q2 = mq_open( QNAME,O_CREAT|O_RDWR,0666,&qattr);  

   if (q2==(mqd_t)-1){      
      int myerrno = errno;

      printf("Errno = %d\n",myerrno);
      if (myerrno == EACCES)
         printf("EACCES\n");
      perror(strerror(myerrno)); 
      assert("Queue opening for creation faliure" == NULL); 
   }
   sleep(3);
   printf("Queues created\n"); 

   assert (pthread_create(&T1_Thid, NULL, thread1, 0) == 0); 
   assert (pthread_create(&T2_Thid, NULL, thread2, 0) == 0); 

   printf("Root started\n"); 
   for (loop=0;loop<10;loop++) {
      printf("Root 2s bling!!!!!!!!!!!!!!!!!\n");      
      sleep(2); 
   } 
   printf("Root is done\n");

   printf("Closing queue handle. Both threads should still work <------------\n");  
   assert (mq_close(q2) == 0);
   sleep(10);

   printf("Unlinking queue name. Threads should block now <------------------\n");    
   assert(mq_unlink(QNAME) == 0);
   sleep(5);

   printf("Hmm, does Linux really following standard? <----------------------\n");  
   sleep(5);

   printf("Killing thread 3\n");  
   assert (pthread_cancel(T3_Thid) == 0);

   printf("Killing thread 2\n");  
   assert (pthread_cancel(T2_Thid) == 0);
   return 0;
} 
\end{verbatim}
\caption{Main program - queues example.\label{q_main}}
\end{table}
%------------------------------------------------------------------------------
%------------------------------------------------------------------------------
%------------------------------------------------------------------------------
\chapter{Master-worker programming model - sorting}
This example implements quick-sort as a threaded operation in a master-worker programming model. In principle it's the same algorithm as the text-book qsort, except that instead of using recursion new threads are started.

The principle is that each word in the text will get it's own thread. This thread only job is to move the word into position and then exit.

It's a silly way to implement qsort on a single-core system\footnote{On a multi-core system it could however actually be quite effective!}. The example only serves the purpose of exemplifying master-worker threading model

Depending on how balanced the text to be sorted is, the number of worker threads can be up to $(2 * n) - 1$, where $n$ is the number of words in the text to be sorted\footnote{Be carefull about the stack usage!}.

\begin{table}[!hbp]
\begin{verbatim}
#define MAX_WORDS 500
#define MAX_SORT_ELEMENT_SIZE 255  //!< Limitation of each sorting elements size

typedef int my_comparison_fn_t (  
   const void *L,  //!< <em>"Leftmost"</em> element to compare with
   const void *R   //!< <em>"Rightmost"</em> element to compare with
);

typedef my_comparison_fn_t *cmpf_t;

typedef struct myarg_t{
   void *a;              //!< The array to be sorted
   int l;                //!< left index
   int r;                //!< right index
   int sz;               //!< Size of each element
   cmpf_t cmpf;          //!< Comparison function 
}myarg_t;

void my_swap (
   void *a,              //!< The array to be sorted
   int l,                //!< left index
   int r,                //!< right index
   int sz                //!< Size of each element   
){
   char t[MAX_SORT_ELEMENT_SIZE];
   memcpy(t,                  &((char*)a)[l*sz],   sz);
   memcpy(&((char*)a)[l*sz],  &((char*)a)[r*sz],   sz);
   memcpy(&((char*)a)[r*sz],  t,                   sz);
}

unsigned int my_qsort_depth = 0; 
unsigned int my_curr_depth = 0;  

#define ISALFANUM( expr ) (\
   (expr >= 'A') && (expr <= 127) ? 1 : 0  \
)

int my_strvcmp (  
   const void *L,
   const void *R 
){
   return strncmp(*(char**)L,*(char**)R,80);   
}
\end{verbatim}
\caption{Structures and data.\label{ccsort_structs}}
\end{table}

\begin{table}[!hbp]
\begin{verbatim}
static const char theText[] = "                                     \
AROUND THE WORLD IN EIGHTY DAYS                                     \
                                                                    \
Chapter I                                                           \
                                                                    \
IN WHICH PHILEAS FOGG AND PASSEPARTOUT ACCEPT EACH OTHER,           \
THE ONE AS MASTER, THE OTHER AS MAN                                 \
                                                                    \
Mr. Phileas Fogg lived, in 1872, at No. 7, Saville Row, Burlington  \
Gardens, the house in which Sheridan died in 1814.  He was one of   \
the most noticeable members of the Reform Club, though he seemed    \
";
\end{verbatim}
\caption{The text to sort. Cut \& paste aprox. 100-300 words from any text of your choice.\label{ccsort_text}}
\end{table}

\begin{table}[!hbp]
\begin{verbatim}
void *conc_quicksort ( void* inarg_vp){
   int i,j;
   void *v;
   struct myarg_t *inarg_p = (myarg_t *)inarg_vp;
   struct myarg_t leftArg;
   struct myarg_t rightArg;
   pthread_t   leftID;
   pthread_t   rightID;

   my_curr_depth++;
   if (my_curr_depth > my_qsort_depth)
      my_qsort_depth=my_curr_depth;

   if ( inarg_p->r > inarg_p->l ){

      v = (void**)(((char*)(inarg_p->a))+(inarg_p->r*inarg_p->sz));
      i = inarg_p->l-1; j = inarg_p->r;
      for (;;){
         while ( inarg_p->cmpf(
            (void**)(((char*)(inarg_p->a))+(++i*inarg_p->sz)), v ) < 0);
         while ( inarg_p->cmpf(
            (void**)(((char*)(inarg_p->a))+(--j*inarg_p->sz)), v ) > 0) ;
         if (i >= j) break;
         my_swap(inarg_p->a,i,j,inarg_p->sz);
      }
      my_swap(inarg_p->a,i,inarg_p->r,inarg_p->sz);

      leftArg.a   = inarg_p->a;
      leftArg.l   = inarg_p->l;
      leftArg.r   = i-1;
      leftArg.cmpf = inarg_p->cmpf;
      leftArg.sz  = inarg_p->sz;

      rightArg.a  = inarg_p->a;
      rightArg.l  = i+1;
      rightArg.r  = inarg_p->r;
      rightArg.cmpf= inarg_p->cmpf;
      rightArg.sz = inarg_p->sz;

      assert (pthread_create(&leftID,  NULL, conc_quicksort, &leftArg)  == 0);
      assert (pthread_create(&rightID, NULL, conc_quicksort, &rightArg) == 0);
      assert (pthread_join(leftID,  NULL) == 0);     
      assert (pthread_join(rightID, NULL) == 0);     
   }
   my_curr_depth--;
   return (void*)0;
}
\end{verbatim}
\caption{Body of each sorting thread (all threads use the same).\label{ccsort_thread}}
\end{table}

\begin{table}[!hbp]
\begin{verbatim}
int main(char argc, char **argv)
{ 
   pthread_t         sortID;
   unsigned int      i,j,k;
   int               inText = 0;
   struct myarg_t    sortArg;

   //Buid up the pointer table
   for (i=0,j=0;i<sizeof(theText);i++){
      assert (j<MAX_WORDS);
     if ( (inText == FALSE) && ISALFANUM(theText[i]) ){
         inText = 1;
         text_p[j] = (char *)&theText[i];
         j++;
      }

      if ( (inText == TRUE) && !ISALFANUM(theText[i]) ){
         inText = 0;
      }
   }

   sortArg.a   = text_p;
   sortArg.l   = 0;
   sortArg.r   = j-2;
   sortArg.sz  = sizeof(char*);
   sortArg.cmpf= my_strvcmp;

   assert (pthread_create(&sortID, NULL, conc_quicksort, &sortArg) == 0);
   assert (pthread_join(sortID, NULL) == 0);

   for (i=0,k=0;i<j;i++){
      assert (j<5000);
      for (k=0;ISALFANUM(text_p[i][k]);k++)
         putchar(text_p[i][k]);
      printf("\n");
   }
   return 0;
}
\end{verbatim}
\caption{Main program for the sorting example.\label{ccsort_main}}
\end{table}

%------------------------------------------------------------------------------
%------------------------------------------------------------------------------
%------------------------------------------------------------------------------
\chapter{Conditional variables}
A text-book example of how to use conditional variables

Two threads are updating the shared resource "count". When count reaches
the agreed upon threshold, a third thread acting as "listener" (and who
is blocked at that time) will be signalled and awakened.

\begin{table}[!hbp]
\begin{verbatim}
int count = 0;
pthread_mutex_t count_mutex = PTHREAD_MUTEX_INITIALIZER;
pthread_cond_t  count_threshold_cv = PTHREAD_COND_INITIALIZER;
#define WATCH_COUNT 12
#define TCOUNT 10

void *watch_count(void *idp){
   pthread_mutex_lock(&count_mutex);
   while (count < WATCH_COUNT){
      pthread_cond_wait(&count_threshold_cv, &count_mutex);
      printf("watch_count(): Thread %d, Count is %d\n", *(int*)idp, count);
   }
   pthread_mutex_unlock(&count_mutex);
   printf("watch_count(): Thread %d finished!\n", *(int*)idp, count);
   return (void*)0;
}
\end{verbatim}
\caption{Listener thread.\label{cond_watcher}}
\end{table}

\begin{table}[!hbp]
\begin{verbatim}
void *inc_count(void *idp){
   int i;

   for (i=0; i<TCOUNT; i++){
      pthread_mutex_lock(&count_mutex);
      count++;
      printf("inc_count(): Threads %d, old count %d, new count %d\n",
         *(int*)idp, count -1, count);
      if (count == WATCH_COUNT)
            pthread_cond_signal(&count_threshold_cv);
      pthread_mutex_unlock(&count_mutex);
      usleep(*(int*)idp * 700 + 1000000);
   }
   printf("inc_count(): Thread %d finished!\n", *(int*)idp, count);
   return (void*)0;
} 
\end{verbatim}
\caption{The writer threads.\label{cond_writer}}
\end{table}

\begin{table}[!hbp]
\begin{verbatim}
int thread_ids[3] = {0, 1, 2};
pthread_t            threads[3]; 
int main(char argc, char **argv)
{ 
   int i;

   printf("Hello world or TinKer targets\n");
   printf("Root started\n");

   assert (pthread_create(&threads[2], NULL, watch_count, &thread_ids[2] ) == 0);
   assert (pthread_create(&threads[0], NULL, inc_count, &thread_ids[0] ) == 0);
   assert (pthread_create(&threads[1], NULL, inc_count, &thread_ids[1] ) == 0);

   for (i=0; i<3; i++){
      assert (pthread_join(threads[i], NULL) == 0);
   }      
   return 0;   
}
\end{verbatim}
\caption{Main program - Conditional variables.\label{cond_main}}
\end{table}

%------------------------------------------------------------------------------
%------------------------------------------------------------------------------
%------------------------------------------------------------------------------
\chapter{Read-write locks}
A text-book example of how to use RW locks
\\\\
Three threads are reading a resource and two threads are writing it

\begin{itemize}
	\item Readers read more often then the writers write (which would be the most common real case).
	\item Readers will not block each other, but would block a writer
	\item A writer will block readers and writers
	\item When releasing lock $\Rightarrow$ If writer blocks readers and writers of same priority, release writer first (part of standard)
	\item When taking read lock $\Rightarrow$ If a writer is blocked and of higher priority than you, back of and let the writer have a chance on the lock.
\end{itemize}

\begin{table}[!hbp]
\begin{verbatim}
int count = 0;
#if defined(linux) &&  (!defined(TINKER))
struct pthread_rwlock_t_ *rw_lock;
#else
pthread_rwlock_t rw_lock = PTHREAD_RWLOCK_INITIALIZER;
#endif
#define TCOUNT 10

void *consumer(void *inpar){
   int done=0;
   int myCount=0;

   printf("<--Consumer thread [%d] starts\n",*(int*)inpar);
   while (myCount<(TCOUNT*2)){
      pthread_rwlock_rdlock(&rw_lock);
         printf("<--Consumer thread [%d] starts reading resource\n",
            *(int*)inpar);
         //Simulate slow access to a resource
         usleep(*(int*)inpar * 300 + 1000000);
         myCount = count;
         printf("<--Consumer thread [%d] finished reading resource.");
         printf("Value is= %d\n",*(int*)inpar,myCount);
      pthread_rwlock_unlock(&rw_lock);
      usleep(*(int*)inpar * 300 + 1000000);
   }
   printf("<--Consumer thread [%d] exits\n",*(int*)inpar);
   return (void*)0;
}

\end{verbatim}
\caption{Reader thread(s).\label{rwl_threads1}}
\end{table}

\begin{table}[!hbp]
\begin{verbatim}
void *producer(void *inpar){
   int i;
   printf("-->Producer thread [%d] starts\n",*(int*)inpar);
   for (i=0; i<TCOUNT; i++){
      pthread_rwlock_wrlock(&rw_lock);
         printf("-->Producer thread [%d] starts updating resource\n",
            *(int*)inpar);
         //Simulate slow access to a resource
         usleep(*(int*)inpar * 700 + 1000000); 
         count++;
         printf("-->Producer thread [%d] finished updating resource.); 
         printf("Value is= %d\n",*(int*)inpar,count);
      pthread_rwlock_unlock(&rw_lock);
      usleep(*(int*)inpar * 700 + 1000000);
   }
   printf("-->Producer thread [%d] exits\n",*(int*)inpar);
   return (void*)0;
} 

\end{verbatim}
\caption{Writer threads(s).\label{rwl_threads2}}
\end{table}


\begin{table}[!hbp]
\begin{verbatim}
int thread_ids[5] = {0, 1, 2, 3, 4};
pthread_t            threads[5]; 
int main(char argc, char **argv)
{ 
   int i;

   #if defined(linux) &&  (!defined(TINKER))
   pthread_rwlock_init(&rw_lock,NULL);
   #endif

   //Creating the reader threads
   assert (pthread_create(&threads[0], NULL, consumer, &thread_ids[0] ) == 0);
   assert (pthread_create(&threads[1], NULL, consumer, &thread_ids[1] ) == 0);
   assert (pthread_create(&threads[2], NULL, consumer, &thread_ids[2] ) == 0);

   //Creating the writer threads
   assert (pthread_create(&threads[3], NULL, producer, &thread_ids[3] ) == 0);
   assert (pthread_create(&threads[4], NULL, producer, &thread_ids[4] ) == 0);

   for (i=0; i<5; i++){
      assert (pthread_join(threads[i], NULL) == 0);
   }
   return 0;
}

\end{verbatim}
\caption{Main program - R/W locks example.\label{rwl_main}}
\end{table}


%------------------------------------------------------------------------------
%------------------------------------------------------------------------------
 % Using a pTread, mqueue based kernel
\part{Real-time}
\chapter{Introduction}
\textit{"It's so fast it has to be \textbf{real-time}\ldots"}
\\\\
\textit{"What I see is on screen is happening in "reality" too, therefor it has to be \textbf{real-time}\ldots."}
\\\\
The above are a indications of what real-time is commonly believed to mean for many people. Of course this is inaccurate.
\\\\
The word real-time is highly misused. Not even Wikipedia has a good definition. I will try to give a good explanation. One of the most competent authors on the subject is in my opinion Philip A Laplante. His textbook  (FIXME: make a reference) is used as an introductory literature in post- \& under-graduate courses at certain universities, and has one of the best and most accurate descriptions.


\chapter{Definition}
A give a very very short description what the concept consists of:
\begin{itemize}
\item Functional determinism
\item Temporal determinism
\end{itemize}
Most so called "gurus" tend to over emphasise on either one and the other and fight bitterly between themselves, then in fact both is true - and to aid sorrow to pain. It's still true in a system that "has" more or less of one or the other. The meaning of real-time has a strong coupling toward the world of automatic control, controlling using computers \& software and controllers. The key-word here in case you missed it is "control" - but also implicitly "to be IN control". I.e. you're the top dog and independent of other unknowns, and you "know" it...


\chapter{Systems vs kernel}
THERE IS A BIG DIFFERENCE BETWEEN A "REAL-TIME SYSTEM" AND A "REAL-TIME KERNEL"
\\\\
A \textit{system} means something that in some way or another consists of (more or less) autonomous \textit{parts}. It indicates some sort of wholeness. In a computer science context we usually mean either a total application consisting of software application, operating system, kernel and computer hardware. Sometimes the word \textit{system} is meant as a short for \textit{"operating system"}, which is actually also true since a \textit{"operating system"} also contains \textit{parts}. 

This chapter aims to convince you of a point that's particulate applicable to the term \textit{real-time}: \textit{A chain is not stronger than it's weakest link.}
\\\\
It doesn't matter wether we mean a operating system or a complete system. If real-time is what we want in the end, the links being represented by the various parts and the chain being the system itself, it will break if somewhere on the line some link doesn't comply with it's real-time criteria. This seems really obvious, yet even professionals manage to mess up here. Misstakes based on not fully understanding the implications of this concept are also unfortunately very common\ldots As a rule of thumb, when ever you think of \textit{real-time}, think of it in the perspective of the whole system. If you do that, the rest will fall in place naturally\ldots
\\\\
Just having a real-time kernel doesn't automatically give you a real-time system (i.e. you're application is the system, or at leas a important part of since we're all talking about embedded applications here). On the other hand, if you aim for a real-time system, you must have a real.time kernel BUT make sure you understand which part of the concept that is essential for your application. 

There is a HUGE difference between various vendors and concepts, and making a mistake choosing the wrong kernel often a \textit{BIG} and practically \textit{irreversible} mistake. And as if that is not enough, choosing an obscure kernel that seems to be right doesn't often shows itself to be the wrong choice until very far into your project, when politics has started to play a role... You don't want to do these mistakes - if you don't feel certain about this and feel pressure you have to contribute with critical decisions without knowing what you're doing - \textbf{RUN}\ldots
\\\\
Note that real-time is a very dictative term and you should carefully explore if you're application needs it of not. Quite often, when people say they need \textit{real-time}, the actually mean something else. This could be either an embedded applications or what-not. However, people not knowing what it stands for use it with a certainly that could lead to (and very often does) that you as a developing engineer base your design on the wrong criteria. Whenever you hear some-one using this term, and especially in the cases where you are supposed to implement that what is communicated, make sure that both you and the originator \textit{mean the same thing}. The point is, \textit{real-time} is something not meant to be taken light upon. It's a very dictative subject and can be a hard, and it really deserves you to treat it respectfully. If you don't, I will most likely come back hard at you\ldots



\chapter{What it's not}
Here comes some commonly used misinterpretations. The following interpretations are \textit{really wrong} and goes far beyond the quarrels between the "gurus" previously mentioned.

\begin{itemize}
\item Real-time video - meaning the opposite of pre-recorded and then played-back video. \textbf{NONSENSE!}
\item  Real-time audio - meaning the opposite of pre-recorded and then played-back audio. \textbf{NONSENSE!}
\item  Real-time stock-exchange - I honestly don't even know what those guys think that they mean by this. Maybe the fact that you can do the exchange over distributed sites... (doh)? Possibly that one has to guarantee that all sites have equal technical ability to make the trade - here's possibly where the concept of "time" comes in - but honestly: Serious database engines has done this for decades, and they still don't call it real-time. \textbf{NONSENSE!}
\item  It's fast - it's in fact so fast that it's \textit{real-time}. As opposed to what? \textit{"Warp speed"} might be a better terminology for this \textbf{NONSENSE!}
\item  And variations of the above, concerning cellular phones voice transfer... GPS map trackinge.t.a..
\end{itemize}
Some people will probably blow a gasket when I start questioning the telecom industry's definition of real-time. However, consider the following:

\begin{itemize}
\item  What is an audio/video stream - It's a flow of samples separated in time.
\item  What in the this has any implication "on real-time" - the fact that jitter will give poor voice quality (jitter will usually introduce so called \textit{"white noise"}). 
\end{itemize}
Note that even a system with jitter \textit{could} fulfil the real-time criteria, if the jitter is known and deterministic. Minimising jitter by making transfer fast and safe (and then optionally add well-known delays) is a way to achieve a certain aspect of real-time, but really: this an academic discussion. My stand point is clear - this is not real-time, the fact that you relay on \textit{statistics} to "prove" anything is \textit{not} the same as being deterministic - no matter how low the probability is, as soon as it's \textit{above zero} it's \textit{not deterministic in definition}. It breaks the rules of being in control, since you're in fact dependant of that never so small probability of the "unknown" to happen\ldots


\paragraph{Alternatives}
Among certain better enlightened people in the telecom industry, another better word is commonly used instead: QoS (or quality of service)  - it says basically the same thing as the telecom people mean, but it's accurate and it's a term not already in use by human-kind meaning something else. Admittedly, the word \textit{real-time} is a catchy word\ldots

\chapter{Functional determinism}
Functional determinism is one of Laplante's two characteristics defining a real time system. It doesn't use the concept of time. Instead it uses a special aspect of time called \textit{events}. 

The reason I mention this aspect of real-time before the temporal one, is that in most cases when you actually need your application to be real-time, this is what you need and will be able to \textit{"get away with"}. Why did I just say \textit{"get away with"}? It's because it's from a kernel and OS perspective much easier to implement and in many cases vendors that call their product "real-time", implement only this aspect. Don't worry though, this is most likely what you need. Real-time in temporal space is actually quite uncommon for most applications.

You could think of events as \textit{markers} in the time continuum. The special thing about functional determinism is that the \textit{actual time for each even is not important}, only the \textit{order in which they arrive} is.
Also needed in the concept is the formal proof. The most commonly used way to proof the systems functional determinism is using "finite automata" (state-charts or state diagrams). Another term commonly used but meaning is FSM (finite state machine). The \textit{BIG} point with state-mashines is that you can proof their \textit{"function"} formally.

State diagrams or state chats also happen to be part of some of the most common description notations. In UML V2 there is a special "aspect" (usually referred to as program behaviour) that uses state charts. Some years ago notation would probably be in state diagrams as defined by Yourdon, Heatley and Phirby 

There exists a lot of literature concerning FSM. Much original work was made by Moore (FIXME: ref) and Mealy (FIXME: ref). David Harel elaborated FSM into so called "state charts" that adds some important benefits to FSM (hierarchic and parallel FSM's).

No matter if you use FSM's in a general spoken context, reading about it or are actually using a formal tool, notice that implicitly "system wide" perspective is always assumed. Even in the case of state-charts, deep down in the hierarchy - that part of the FSM could not exist in reality out of context.

FSM's are really nothing to me intimidated by. If you're not into implementing methods for formal proof, they're easy as pie:

\begin{dotpic}
   node [
      shape=record,
      style=filled,
      fillcolor=yellow,
   ];

   edge [
      color="red",
      fontname=Courier,
      justify="center",
      fontsize=10.0
   ];

   graph [
      rankdir = "TB",
      fontname=Courier,
      nojustify="true",
      fontsize=10.0
   ];

   S[
      shape=circle,
      style=filled,
      fillcolor=black
   ];

   On [ orientation=73.0, label="{\
     Entry:  Turn_on_lights()       | \
     Entry:  Set_speed_idle()       | \	
      --- ON ---  | \
     Exit:  Turn_off_lights() }"];

   StdBy [ orientation=73.0, label="{\
     Entry:  Set_speed_idle()       | \
      --- STDBY ---  | \
     Exit: }"];

   Off[
      shape=circle,
      style=filled,
      fillcolor=white

   ];
   

    S -> On    [label="MainSwitch==ON"]
    S -> StdBy [label="MainSwitch==STDBY"]
    On -> StdBy [label="MainSwitch==STDBY"]
    StdBy -> On [label="MainSwitch==ON"]
    StdBy -> Off [label="MainSwitch==OFF [Set_speed(0) / Turn_off_lights()]"]

\end{dotpic}
 
The figure above is a silly little example in Harel notation. In reality a state-chart can be much more complex, covering quite a few pages with system wide charts, sub-system charts, sub-charts, sub-chart of sub-charts and so on\ldots However, the big point of using FSM is that's easy and that the design is functionally proffable. It's without competition the easiest and most intuitive way to describe a application interned for some kind of control available. It's a big shame that this is not recognised by everyone. Instead some projects try to describe their mashines in notation much better suited for web-pages or desktop applications. These tools are certainly also good for what they are intended, but use the right methodology (i.e. inherently also the right notation or descriptive language) for your problem. Some tools are as ill suited for control applications as a screw-driver is to a bricklayer!

\chapter{Temporal determinism (or determinism in temporal space)}
This is a topic most commonly associated with real-time by either the general public or by some certain specialists.

Temporal determinism actually \textit{has} the notion of time built in. In short what it means is that the system has to respond to an event withing a certain time frame. 

This can be expressed as that the system has to respond \textit{faster} than a certain elapsed time, but it can also mean that the system must not respond until a certain amount of time has elapsed (i.e it must respond \textit{slower} than a certain time). It can also mean a combination of these two. I.e. response has to come after a certain elapsed time, but before a certain other elapsed time.

I.e. the system handling temporal determinism requirements has to have knowledge of time 

Per definition, what it also means, is that any requirement expressed as any of the above must be \textbf{\textit{}proofable}, and he's where're the infected debate starts\ldots

It turns out that this is in practical terms quite difficult to do. There exists formal methods doing it (rate monotonic analysis, petri nets e.t.a.) but the main problem is not the formal proof, it's the implementation! He're is a list of the complicating factors:

\begin{itemize}
\item Many OS vendors don't recognise temporal determinism, even those who call themselves RTOS 
\item Real-time is inherently a system-wide aspect that plays a role in the system and this often clashes with the next point.
\item Quite often, what is considered "valuable" are features that have nothing to do with real-time at all. The ability to remotely control an application over a network, most user interfaces to name but a few\ldots However, those features you really don't want to implement from scratch by youself (i.e. from chip drivers, protocols, stacks and application). It's just way too much work! If you're sane you'd use an operating system that provides you with support for these things. However, most of them doe's not handle temporal determinism.
\item Those operating systems or kernels that actually does handle temporal determinism are very few, and most of them do not address the goodies mentioned above at all (or quite badly).
\end{itemize}

Very common is that a system has \textit{some} truly temporal requirements, but also quite a few (or most even) not belonging to the temporal domain at all. These systems are sometimes referred to as hybrid systems. The term is really unnecessary, but it makes a point of the distinction between two really importand factors each with very strong implications in a control system. 
The key issue is proof. Some people argue that the only way to implement a software system having true temporal requirements is to make it as the silly little example in our chapter "A quick tour" - and some vendors actually does exactly that. Even though this is formally correct, the project looses enormous potential, by at the same time disregard all other abilities. Yes, they can be implemented on top of your application - but is is sane to do it? In a project with unlimited budget and with absolute safety requirements - perhaps. For all other normal projects this would be a bad idea.
\\\\
The other way to relate to this problem is equally bad, however it's the most common one. The key issue here is to make a small adjustment to the real-time temporal definition itself by allowing a certain flexibility in the timely requirements. This is what is meant by \textit{"hard real-time"} and \textit{"soft real-time"}. Hard real-time being the original strict definition and soft real-time the relaxed one.

The basic idea is to relax the proofability of the requirement. Or by requiring response times to follow a certain distribution. This is all good in theory. The problem is that this is even harder to do! It is immensely difficult to make a tools or system that can proof statistical distribution. And even if there does indeed exist a few claiming being able doing so, how do you proof that these measuring or analysing what they claim. 

It's actually quite easy to come quite close \textit{"reasonable"} certainty, but it's very very difficult to come really close to it.

Since I don't believe the matter can be handled properly and since I don't trust any of the product's claiming to be able to do it. My standpoint is clear: s this is not real-time and should be banned!
\\\\
But then all that remains is the first method with all it's inbound disadvantages then? \textit{WRONG!} There exists yet an alternative\ldots

\chapter{Real real-time - the easy way}
Common sense is a wildly underestimated ability among computer programmers computer ingeners. It's also something not quite realized as a real problem by the scientific community. This is truly sad because it causes a lot of unnecessary frustration and pain\ldots
\\\\
The \textit{easy way} to handle the issue involved with hybrid systems are actually just that: easy! 

Some rules of thumb:
\begin{itemize}
\item Ask youself if you system really is a hybrid system or not. It's not unlikely that you don't have requirements in the temporal space even if you think you do, Remember: writing a fantastic requirement is easy, implementing it is something else. For example, when an mechanical engineer makes a drawing, he always adds tolerances to all heäs measures. This discipline allready knows that there are no such thing as absolute accuracy, but that you can come close. The tolerance is a sort of requirement for the manufacture, and will determine which manufacturing process to use (i.e. inherently how mush it will cost). Generally speaking, the more accurate the higher the manufacturing price.
\item Suppose you really do have a hybrid system. Try to implement the temporal requirements in a autonomous sumsystem of it's own. That way you can use a methodology suited for that domain but not paying the price for it. Key issue here is \textit{separation of requirements} in dedicated problem domains who in turn addresses them with methods best suited.
\item Long experience has thought me that in hybrid system, the ratio between requirements in the temporal domain and other requirements usually has a big ration. I.e. there are generally many more non-temporal requirements than temporal. Not unusual are having less than a handfull temporal requirements, but several hundred (or even thousands) of non temporal.
\item Do you really have to handle the temporal parts in software. Not unusual is that the requirement can be handled by some external (and often quite simple) electronic device separated from the CPU system.
\end{itemize}



And finally, if you \textit{really} have explored all alternatives and \textit{really} have to implement temporal requirements in software, make a very carefull choice. Do not get fooled to trade away one ability on the behalf of another and \textit{DO NOT LISTEN TO SALES PERSONS}. The reason is that they quite often don't know what they're talking about even tough the think they do (tough subject, ouch!, Don't even go there\ldots). I.e. they might actually fool you without even knowing it.

Do not choose a RTOS or RT kernel trading away portability. If you trade away mumbo-jumbo stuff like networking and graphical fancy staff, fine. But at least don't trade away portability and transparency! It's very important that the design you make is not dependant of one certain vendor or you will put you project and perhaps even the whole company in a very big risk. What if that RTOS is revoked from the market? This is not very uncommon and depending on the lifespan of you product this might be either a non issue or a show-stopper. Having worked in the medical devices industry for the last 10 years with product lifespans well over 15 years (not uncommon is product life span yell over 25 years!), you really don't want to pot yourself in a position where you can't harvest the fruit of your invention only because some company made a hostile takeover (or something similar\ldots).

In my opinion, the only way to handle this considering all of the issues mentioned are to\ldots
\begin{itemize}
\item Choose a kernel compliant with POSIX standard (or the sub.set of POSIX that you need). This will allow you to escape from the vendor and also creates all sorts of other positive financial advantages. Be very determined when you choose and again \textit{DON'T LISTEN TO SALES PERSONS}, at least not to much\ldots
\item Choose a vendor with proven reputation or someone willing to sign compliance of the safety regulations applicable to you.
\item Use a \textit{\textbf{free kernel}}. If you find youself in the position that you have to formally verify or validate a kernel yourelf, you might as well have the source code for it!
\end{itemize}
 % Real-time
%------------------------------------------------------------------------------

\listoffigures 
\listoftables
\end{document}
