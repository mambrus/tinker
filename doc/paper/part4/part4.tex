\part{Real-time}
\chapter{Introduction}
\textit{"It's so fast it has to be \textbf{real-time}\ldots"}
\\\\
\textit{"What I see is on screen is happening in "reality" too, therefor it has to be \textbf{real-time}\ldots."}
\\\\
The above are a indications of what real-time is commonly believed to mean for many people. Ofcause it is inaccurate.
\\\\
The word real-time is highly misused. Not even Wikipedia has a good definition. I will try to give a good explanation. One of the most competent authors on the subject is in my opinion Philip A Laplante. His textbook  (FIXME: make a reference) is used as an introductory literature in post- \& under-graduate courses at certain universities, and has one of the best and most accurate descriptions.


\chapter{Definition}
A give a very very short description what the concept consists of:
\begin{itemize}
\item Functional determinism
\item Temporal determinism
\end{itemize}
Most so called "gurus" tend to over emphasize on either one and the other and fight bitterly between themselves, then in fact both is true - and to aid sorrow to pain. It's still true in a system that "has" more or less of one or the other. The meaning of real-time has a strong coupling toward the world of automatic control, controlling using computers \& software and controllers. The key-word here in case you missed it is "control" - but also implicitly "to be IN control". I.e. you're the top dog and independent of other unknowns, and you "know" it...


\chapter{Systems vs kernel}
THERE IS A BIG DIFFERENCE BETWEEN A "REAL-TIME SYSTEM" AND A "REAL-TIME KERNEL"
\\\\
A \textit{system} means something that in some way or another consistes of (more or less) autonomus \textit{parts}. It indicates some sort of wholeness. In a computer scinense context we usually mean either a total application consisting of software application, operating system, kernel and computer hardware. Sometimes the word \textit{system} is ment as a short for \textit{"operating system"}, which is actually also true since a \textit{"operating system"} also contains \textit{parts}. 

This chapter aims to convince you of a point that's particulaty applicable to the term \textit{real-time}: \textit{A chain is not stonger than it's weakest link.}
\\\\
It doesn't matter wether we mean a operating system or a complete system. If real-time is what we want in the end, the links being represented by the various parts and the chain beeing the system itself, it will break if somewhere on the line some link doesn't compy with it's real-time criteria. This seems really obvious, yet even proffesionals manage to mess up here. Misstakes based on not fully understanding the implications of this concept are also unfortunatly very common\ldots As a rule of thumb, when ever you think of \textit{real-time}, think of it in the perspective of the whole system. If you do that, the rest will fall in place natuarally\ldots
\\\\
Just having a real-time kernel doesn't automatically give you a real-time system (i.e. you're application is the system, or at leas a important part of since we're all talking about embedded applications here). On the other hand, if you aim for a real-time system, you must have a real.time kernel BUT make sure you understand which part of the concept that is essential for your application. 

There is a HUGE difference between various vendors and concepts, and making a mistake choosing the wrong kernel often a \textit{BIG} and practically \textit{irreversible} mistake. And as if that is not enough, choosing an obscure kernel that seems to be right doesn't often shows itself to be the wrong choice until very far into your project, when politics has started to play a role... You don't want to do these mistakes - if you don't feel certain about this and feel pressure you have to contribute with critical decisions without knowing what you're doing - \textbf{RUN}\ldots
\\\\
Note that real-time is a very dictative term and you should carefully explore if you're application needs it of not. Quite often, when people say they need \textit{real-time}, the actually mean something else. This could be either an embedded applications or what-not. However, peole not knowing what it stands for use it with a certany that could lead to (and very often does) thjat you as a developing engieer base your design on the wring criterial. Whenever you hear some-one using this term, and especially in the cases where you are supposed to implement that what is communiocated, make sure that both you and the originator \textit{mean the same thing}. The point is, \textit{real-time} is something not ment to be taken light upon. It's a very dictative subject and can be a hard, and it really desevers you to treat it respectfully. If you don't, I will most likely come back hard at you\ldots



\chapter{What it's not}
Here comes some commonly used misinterpretations. The following interpretations are \textit{really wrong} and goes far beyond the quarrels between the "gurus" previously mentioned.

\begin{itemize}
\item Real-time video - meaning the opposite of pre-recorded and then played-back video. \textbf{NONSENSE!}
\item  Real-time audio - meaning the opposite of pre-recorded and then played-back audio. \textbf{NONSENSE!}
\item  Real-time stock-exchange - I honestly don't even know what those guys think that they mean by this. Maybe the fact that you can do the exchange over distributed sites... (doh)? Possibly that one has to guarantee that all sites have equal technical ability to make the trade - here's possibly where the concept of "time" comes in - but honestly: Serious database engines has done this for decades, and they still don't call it real-time. \textbf{NONSENSE!}
\item  It's fast - it's in fact so fast that it's \textit{real-time}. As opposed to what? \textit{"Warp speed"} might be a better terminology for this \textbf{NONSENSE!}
\item  And variations of the above, concerning cellular phones voice transper... GPS map trackinge.t.a..
\end{itemize}
Some people will probably blow a gasket when I start questioning the telecom industry's definition of real-time. However, consider the following:

\begin{itemize}
\item  What is an audio/video stream - It's a flow of samples separated in time.
\item  What in the this has any implication "on real-time" - the fact that jitter will give poor voice quality (jitter will usually introduce so called \textit{"white noice"}). 
\end{itemize}
Note that even a system with jitter \textit{could} fulfill the real-time criteria, if the jitter is known and deterministic. Minimizing jitter by making transfer fast and safe (and then optionally add well-known delays) is a way to achieve a certain aspect of real-time, but really: this an academic discussion. My stand point is clear - this is not real-time, the fact that you relay on \textit{statistics} to "prove" anything is \textit{not} the same as being deterministic - no matter how low the probability is, as soon as it's \textit{above zero} it's \textit{not deterministic in definition}. It breaks the rules of being in control, since you're in fact dependant of that never so small probability of the "un-known" to happen\ldots


\paragraph{Alternatives}
Among certain better enlightened people in the telecom industry, another better word is commonly used instead: QoS (or quality of service)  - it says basically the same thing as the telecom people mean, but it's accurate and it's a term not already in use by human-kind meaning something else. Admittedly, the word \textit{real-time} is a catchy word\ldots

\chapter{Functional determinism}
Functional determinism is one of Laplantet's two characteristics defineing a real time system. It doesn't use the concept of time. Instead it uses a special aspect of time called \textit{events}. 

The reason I mention this aspect of real-time befor the temopral one, is that in most cases when you actually need your application to be real-time, this is what you need and will be able to \textit{"get away with"}. Why did I just say \textit{"get away with"}? It's because it's from a kernel and OS perspective much easier to implement and in many cases vendors that call their product "real-time", implement only this aspect. Don't worry though, this is most likely what you need. Real-time in temporal space is actually quite uncommon for most applications.

You could think of events as \textit{markers} in the time continuum. The special thing about functional determinism is that the \textit{actual time for each even is not important}, only the \textit{order in which they arrive} is.
Also needed in the concept is the formal proof. The most commponty used way to proof the systems functional determinism is using "finite automata" (state-charts or state diagrams). Another term commonly used but meaning is FSM (finite state mashine). The \textit{BIG} point with state-mashines is that you can proof their \textit{"funtion"} formally.

State diagrams or state chats also happen to be part of some of the most common descrition notations. In UML V2 there is a special "aspect" (usually refered to as program behaviour) that uses state charts. Some years ago notation would probably be in state diagrams as defined by Yourdon, Heatley and Phirby 

There exists a lot of litterature concering FSM. Much original work was made by Moore (FIXME: ref) and Mealy (FIXME: ref). David Harel elaborated FSM into so called "state charts" that adds some important benefits to FSM (hieracical and paralell FSM's).

No matter if you use FSM's in a genaral spoken context, reading about it or are actually using a formal tool, notice that implicitly "system wide" pespective is always assumed. Even in the case of state-charts, deep down in the hiercy - that part of the FSM could not exist in reality out of context.

FSM's are really nothing to me intimidated by. If you're not into implementing methods for formal proof, they're easy as pie:

\begin{dotpic}
   node [
      shape=record,
      style=filled,
      fillcolor=yellow,
   ];

   edge [
      color="red",
      fontname=Courier,
      justify="center",
      fontsize=10.0
   ];

   graph [
      rankdir = "TB",
      fontname=Courier,
      nojustify="true",
      fontsize=10.0
   ];

   S[
      shape=circle,
      style=filled,
      fillcolor=black
   ];

   On [ orientation=73.0, label="{\
     Entry:  Turn_on_lights()       | \
     Entry:  Set_speed_idle()       | \	
      --- ON ---  | \
     Exit:  Turn_off_lights() }"];

   StdBy [ orientation=73.0, label="{\
     Entry:  Set_speed_idle()       | \
      --- STDBY ---  | \
     Exit: }"];

   Off[
      shape=circle,
      style=filled,
      fillcolor=white

   ];
   

    S -> On    [label="MainSwitch==ON"]
    S -> StdBy [label="MainSwitch==STDBY"]
    On -> StdBy [label="MainSwitch==STDBY"]
    StdBy -> On [label="MainSwitch==ON"]
    StdBy -> Off [label="MainSwitch==OFF [Set_speed(0) / Turn_off_lights()]"]

\end{dotpic}
 
The figure above is a silly little example in Harel notation. In reality a state-chart can be much more complex, covering quite a few pages with system wide charts, sub-system charts, sub-charts, sub-chart of sub-charts and so on\ldots However, the big point of using FSM is that's easy and that the design is functionaly proffable. It's without competition the easiest and most intuitive way to describe a application intened for some kind of control available. It's a big shame that this is not recognized by everyone. Instead some projects try to describe their mashines in notation much better suited for web-pages or desktop applications. These tools are certainly also good for what they are intended, but use the right methology (i.e. inherently also the right notation or descriptive language) for your problem. Some tools are as ill suited for control applications as a screw-driver is to a bricklayer!

\chapter{Temporal determinism (or determinsim in temporal space)}
This is a topic most commonly associated with real-time by either the general public or by some certain specialists.

Temporal determinism actually \textit{has} the notion of time built in. In short what it means is that the system has to respond to en event withing a certain time frame. 

This can be expressed as that the system has to respond \textit{faster} than a certain elapsed time, but it can also mean that the system must not respont until a certain amount of time has elapsed (i.e it must respond \textit{slower} than a certain time). It can also mean a combination of these two. I.e. response has to come after a certain elapsed time, but before a cretian other elapsed time.

I.e. the system handling temporal determinism requirements has to have knowledge of time 

Per definition, what it also means, is that any requirement expressed as any of the above must be \textbf{\textit{}proofable}, and he's wher're the infected debate starts\ldots

It turns out that this is in practical terms quite difficult to do. There exists formal methods doing it (rate monotonic analysis, petri nets e.t.a.) but the main problem is not the formal proof, it's the implementation! He're is a list of the complicating factors:

\begin{itemize}
\item Many OS vendors don't recognize temporal determinism, even those who call themselfs RTOS 
\item Real-time is inherently a system-wide aspect that plays a role in the system and this often clashes with the next point.
\item Quite often, what is considered "valuable" are features that have nothing to do with real-time at all. The ability to remotly control an application over a network, moste user interfaces to name but a few\ldots However, those features you really don't wan't to implement from scratch by youself (i.e. from chip drivers, protocols, stacks and application). It's just way too much work! If you're sanem you'd use an operating system that provides you with support for these things. However, most of them doe's not handle temporal determinism.
\item Those operating systems or kernels that actually does handle temporal determinism are very few, and most of them do not adress the goodies mentioned above at all (or quite badly).
\end{itemize}

Very common is that a system has \textit{some} truly temporal requirements, but also quite a few (or most even) not belonging to the temporal domain at all. These systems are sometimes refered to as hybrid systems. The term is really unnessesary, but it makes a point of the distincion bewteen two really importand factors each with very strong implications in a control syste. 
The key issue is proof. Some people argue that the only way to implement a software system having true temopral requirements is to make it as the silly little example in our chapter "A quick tour" - and some vendors actually does exactly that. Even though this is formally correct, the project looses enormous potential, by at the same time disregard all other abilities. Yes, they can be implemented on top of your application - but is is sane to do it? In a project with unlimited budget and with absolute safety requirements - perhaps. For all other normal projects this would be a bad idea.
\\\\
The other way to relate to this problem is equally bad, however it's the most common one. The key issue here is to make a small adjustment to the real-time temporal definition itself by allowing a certain flexibility in the timly requirements. This is what is ment by \textit{"hard real-time"} and \textit{"soft real-time"}. Hard real-time being the original strict definition and soft real-time the relaxed one.

The basic idea is to relax the proofability of the requirement. Or by recuireing response times to follow a certain distribution. This is all good in theory. The problem is that this is even harder to do! It is emensly difficult to make a tools or system that can proofe statistical distribution. And even if there does indeed exist a few claiming beeing able doing so, how do you proofe that these meassuring or analyzing what they claim. 

It's actually quite easy to come quite close \textit{"resonable"} certanty, but it's very very difficult to come really close to it.

Since I don't believe the matter can be handled properly and since I don't trust any of the product's claiming to be able to do it. My standpoint is clear: s this is not real-time and should be banned!
\\\\
But then all that remains is the first method with all it's inbound disadvantages then? \textit{WRONG!} There exists yet an alternative\ldots

\chapter{Real real-time - the easy way}
Common sence is a wildly underestimated ability among computer programmers computer ingeneers. It's also something not quite realized as a real problem by the scientific community. This is trully sad because it couses a lot of unnesesary frustration and pain\ldots
\\\\
The \textit{easy way} to handle the issue involved with hybrid systems are actually just that: easy! 

Some rules of thumb:
\begin{itemize}
\item Ask youself if you system really is a hybrid system or not. It's not unlikly that you don't have requirements in the temporal space even if you think you do, Remeber: writing a fantastic requirement is easy, implementing it is something else. For example, when an mechanical engeneer makes a drawing, he always adds tolerances to all heäs meassures. This dicipline allready know's that there are no such thing as absolute accurancy, but tht you can come close. The tolerance is a sort of requirement for the manufactur, and will determinde which manufacturing process to use (i.e. inherently how mush it will cost). Generally speaking, the more accurate the higher the manufacturing price.
\item Suppose you really do have a hybrid system. Try to implement the temporal requirements in a autonomos sumsystem of it's own. That way you can use a methodology suited for that domain but not paying the price for it. Key issue here is \textit{separation of requirements} in dedicated problem domainsm who in turn addresses them with methods best suited.
\item Long experiance has tought me that in hybrid system, the ratio between requirements in the temporal domain and other requirements usually has a big ration. I.e. there are generally many more non-temporal requirements than temporal. Not unusual are having less than a handfull temporal requirements, but several hundred (or even thousands) of non temporal.
\item Do you really have to handle the temporal parts in software. Not unusual is that the requirement han be handled by some external (and often quite simple) electronic device separated from the CPU system.
\end{itemize}



And finally, if you \textit{really} have explored all alternatives and \textit{really} have to implement temporal requirements in software, make a very carefull choice. Do not get fooled to trade away one ability on the behalf of another and \textit{DO NOT LISTEN TO SALES PERSONS}. The reason is that they quite often don't know what they're talking about even tough the think they do (tough subject, ouch!, Don't even go there\ldots). I.e. they might actually fool you without even knowing it.

Do not choose a RTOS or RT kernel trading away portability. If you trade awya mumbo-jumbo stuff like networking and graphical fancy staff, fine. But at least don't trade away portability and transparence! It's very imoprtant that the design you make is not dependant of one certain vendor or you will put you project and perhaps even the whole company in a very big risk. Wha if that RTOS is revoked from the makert? This is not very uncommon and depending on the lifespan of you product this might be either a non issue or a show-stopper. HAving worked in the medical devices industy for the last 10 yeras wuth product lifespans well over 15 years (not uncommon is product life span yell over 25 years!), you really dont want to pot yourself in a position where you can't harvest the fruit of your invention only because some company made a hostile takeover (or something similar\ldots).

In my opinon, the only way to handle this concidering all of the issues mentioned is to
\begin{itemize}
\item Choose a kernel compliant with POSIX standard (or the sub.set of POSIX that you need). This will allow you to escape from the vendor and also creates all sorts of other positive financial advantages. Be very determined when you choose and again \textit{DON'T LISTEN TO SALES PERSONS}, at least not to much\ldots
\item Choose a vendor with proven reputation or someone willing to sign compience of the safety regulations applicable to you.
\item Use a \textit{\textbf{free kernel}}. If you find youself in the position that you have to formaly verify or validate a kernel yourelf, you might as well have the sorce code for it!
\end{itemize}
